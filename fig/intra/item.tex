\begin{figure}[t]
\centering
\begin{overpic} 
[width=\linewidth]
% [width=\linewidth,grid,tics=10]
{\currfiledir item.pdf}
\myfigurename{}
\put(15,40.5){straight finger}
\put(65,40.5){bent finger}
% 
\put(23,-3){$x_s^*$}
\put(40,-3){$x_s^+$}
\put(7.5,-3){$x_s^-$}
% 
\put(57,-3){$x_b^-$}
\put(74,-3){$x_b^*$}
\put(90,-3){$x_b^+$}
\end{overpic}
\caption{
% 
% 
\textbf{(Intra-frame regression)} We abstract the hand shape/pose estimation problem from a single frame into the one of a simpler 2D stick-figure.  When the finger is straight (left), it is difficult to estimate finger length as the optimization problem is ill-posed, while with a bent finger (right) the problem is better conditioned.
% 
We analyze the landscape of the registration energies {\color{anagreen}$E(\shape_1)$}, and observe how estimation uncertainty relates to the width of the local minima valley. This certainty can be estimated through a quadratic approximation {\color{anasalmon}$\tilde{E}(\shape_1)$}, derived from the Jacobians of the registration energies.
% 
%
}
\label{fig:intra}
\end{figure}
