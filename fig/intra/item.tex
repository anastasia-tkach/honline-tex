\begin{figure}[t]
\centering
\begin{overpic} 
[width=\linewidth]
% [width=\linewidth,grid,tics=10]
{\currfiledir item.pdf}
\myfigurename{}
\put(15,40.5){straight finger}
\put(65,40.5){bent finger}
% 
\put(28.5,9.5){\scriptsize $\beta_{[1]}$}
\put(18,19.5){\scriptsize $\beta_{[2]}$}
% 
\put(23,-3){\scriptsize $\beta_{[1]}^*$}
\put(40,-3){\scriptsize $\beta_{[1]}^+$}
\put(7.5,-3){\scriptsize $\beta_{[1]}^-$}
% 
\put(57,-3){\scriptsize $\beta_{[1]}^-$}
\put(74,-3){\scriptsize $\beta_{[1]}^*$}
\put(90,-3){\scriptsize $\beta_{[1]}^+$}
\end{overpic}
\caption{
% 
%
\textbf{(Intra-frame regression)} We abstract the hand shape/pose estimation problem from a single frame into the one of a simpler 2D stick-figure. Note however our illustration are not hand-crafted, but are derived from numerical optimization executed on these simplified datasets. When the finger is straight (left), it is difficult to estimate finger length as the optimization problem is ill-posed, while with a bent finger (right) the problem is better conditioned.
% 
We analyze the landscape of the registration energies {\color{anagreen}$E(\shape_{[1]})$}, and observe how estimation uncertainty relates to the width of the local minima valley. This certainty can be estimated through a quadratic approximation {\color{anasalmon}$\tilde{E}(\shape_{[1]})$}, derived from the Jacobians of the registration energies.
% 
%
}
\label{fig:intra}
\end{figure}
