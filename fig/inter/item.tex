\begin{figure}[t]
\centering
\begin{overpic} 
[width=\linewidth]
% [width=\linewidth,grid,tics=10]
{\currfiledir/item.pdf}
\myfigurename{}
% 
\put(18.5,41){\scriptsize $\beta_{[1]}$}
\put(56,47){\scriptsize $\beta_{[2]}$}
\put(94,50){\scriptsize $\beta_{[3]}$}
% 
\put(75,46){\scriptsize $\theta_{[2]}$}
\put(37.5,41){\scriptsize $\theta_{[1]}$}
% 
\put(3.5,40){\scriptsize \vertical{input data}}
\put(0,43){\scriptsize \vertical{$d_n$}}
%
\put(3.5,23){\scriptsize \vertical{intra-frame}}
\put(0,22){\scriptsize \vertical{$\mathcal{N}(\star{x}_n, \star{\Sigma}_n)$}}
% 
\put(3.5,4){\scriptsize \vertical{inter-frame}}
\put(0,3){\scriptsize \vertical{$\mathcal{N}(\hat{x}_n, \hat{\Sigma}_n)$}}
% 
\put(12.0,-2){\scriptsize $n=1$}
\put(30.5,-2){\scriptsize $n=2$}
\put(49.5,-2){\scriptsize $n=3$}
\put(68.5,-2){\scriptsize $n=4$}
\put(87.0,-2){\scriptsize $n=5$}
% 
\end{overpic}
\caption{
% 
\textbf{(Inter-frame regression)} We visualize several temporally sorted frames of input data $d_n$, the uncertainty ellipsoid $\star{\Sigma}_n$ estimated by intra-frame regression, and the \emph{online} uncertainty estimate $\hat{\Sigma}_n$. For illustration purposes, we only display the two-dimensional ellipsoids representing the covariance of $\beta_{[1]}$ and $\beta_{[2]}$. Although $\star{\Sigma}_1=\star{\Sigma}_5$, observe how $\hat{\Sigma}_1 \succ \hat{\Sigma}_5$: in the last frame we have a confident estimate as the information from frames {\scriptsize $2:4$} has been integrated.
% 
Further, notice how, even though the parameter $\beta_{[2]}$ was not observed directly in any of the presented frames, its value was inferred from the measurements \AN{high certainty measurements. It is important to say that they are high certainty}of~$(\beta_{[1]})_{n=2}$~and~$(\beta_{[1]} + \beta_{[2]})_{n=3}$.
%; see also \Figure{merging}.
%  
}
\label{fig:inter}
\end{figure}
