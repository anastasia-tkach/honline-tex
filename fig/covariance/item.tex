\begin{SCfigure}[][b]
\centering
\begin{overpic} 
[width=.5\linewidth]
% [width=\linewidth,grid,tics=10]
{\currfiledir/item.pdf}
\myfigurename{}
% \put(0,8){\scriptsize $\beta_2$}
% \put(0,3){\scriptsize $\beta_1$}
% 
\put(93,6){\scriptsize $\theta_{[1]}$}
\put(4,95){\scriptsize $\theta_{[2]}$}
\put(90,-2){\scriptsize $\pi$}
\put(-3,91){\scriptsize $\pi$}
% 
% \put(89,3){\scriptsize $\theta_1$}
% \put(5,89){\scriptsize $\theta_2$}
\end{overpic}
% \vspace{-.05in}
\caption{
% 
A visualization of the covariance estimate for phalanx lengths $\{\beta_{[1]}, \beta_{[2]}\}$ as we vary phalanx bend angles $\{\theta_{[1]},\theta_{[2]}\}$.
% 
Note how a confident measurement 
% \AN{could you please replace guess by measurement or estimate?} \AT{done}
for the $\star\beta$ parameters is only available when $\theta_{[1]}$ and $\theta_{[2]}$ are simultaneously large, that is, with both phalanxes bent.
% 
In this visualization the covariance ellipsoids have been re-centered.
% 
% \AT{should we remove drawings?}
}
\label{fig:covariance}
\end{SCfigure}
