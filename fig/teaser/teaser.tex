\begin{teaserfigure}
\begin{overpic} 
[width=\linewidth]
% [width=\linewidth,grid,tics=10]
{fig/teaser/item.pdf}
\put(3,9){$\color[RGB]{198,94,125} \text{diag}(\star\Sigma)^{-1}$}
\put(3,4){$\color[RGB]{103,177,159} \text{diag}(\hat\Sigma)^{-1}$}
\myfigurename{}
\end{overpic}
\centering
\vspace{-.2in}
\caption{
% 
% 
% 
Our adaptive tracking algorithm optimizes for a tracking model on-the-fly, leading to automatic improvements in  tracking accuracy over time. We color-code the surface to visualize the spatially-varying confidence of the estimated geometry. \todo{In the top-right of each frame we also visualize the template without over-imposing the point cloud.} \AT{should this be visualized in the rest pose?} \AN{In the middle of each frame the model surface is color-coded with the certainties of the parameters measured in that particular frame. In the top-right corner the color-coding corresponds to the estimated certainty of the parameters accumulated since the beginning of the sequence. Notice how at the last frames all parameters are certain}.The histograms visualize the uncertainty of each degree of freedom, that is the diagonal entries of the Hessian estimated from data in the current frame $\star\hessian$, or by accumulating the information through time $\hat\hessian$. These results are best appreciated by watching the supplementary \textbf{video}.
%
%
% 
}
\label{fig:teaser}
\end{teaserfigure}
