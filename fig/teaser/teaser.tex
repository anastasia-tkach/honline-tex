%!TEX root = ../../paper.tex
\begin{teaserfigure}
\begin{overpic} 
[width=\linewidth]
% [width=\linewidth,grid,tics=10]
{fig/teaser/item.pdf}
\put(3,9){$\color[RGB]{198,94,125} \text{diag}(\star\Sigma)^{-1}$}
\put(3,4){$\color[RGB]{103,177,159} \text{diag}(\hat\Sigma)^{-1}$}
\myfigurename{}
\end{overpic}
\centering
\vspace{-.2in}
\caption{
% 
% 
% 
Our adaptive hand tracking algorithm optimizes for a tracking model on the fly, leading to progressive improvements in tracking accuracy over time.   
{\bf Above}: Hand surface color-coded to visualize the spatially-varying confidence of the estimated geometry. Insets: color-coded \emph{cumulative} certainty.  Notice how in the last frame all parameters are certain. 
{\bf Below}: Histograms visualize the certainty of each degree of freedom, that is, the diagonal entries of the inverse of the covariance estimate from:
(a) data in the current frame $\star\Sigma$, or 
(b) the accumulated information through time $\hat\Sigma$. 
% These results are best appreciated by watching the supplementary \textbf{video}.
%
%
% 
}
\label{fig:teaser}
\end{teaserfigure}
