\begin{figure}[t]
\centering
\begin{overpic} 
[width=\linewidth]
% [width=\linewidth,grid,tics=10]
{\currfiledir/item.pdf}
\myfigurename{}
\put(18,-3){Pose DOFs $\theta_{[i]}$}
\put(68,-3){Shape DOFs $\beta_{[i]}$}
\end{overpic}
\caption{
% 
The degrees of freedom of our optimization, where use a cartesian right-handed coordinate frame for translational DOFs. 
For pose parameters, we color code 
{\color[RGB]{53,120,109} flexion or x-translation}, 
{\color[RGB]{212,144,133} twist or y-translation}, and {\color[RGB]{172,72,100} abduction or z-translation}. 
For shape parameters, we color code 
{\color[RGB]{172,72,100} lengths},
{\color[RGB]{53,120,109} \{x,y,z\} positions},
{\color[RGB]{129,190,163} \{x,y\} positions \AT{diff from above?}},
{\color[RGB]{212,144,133} radii}, and
{\color[RGB]{120,120,120} depend on other dofs \AT{what?}}.
% 
\AT{and by the way, how can there be translations in the pose parameters beside global translation?}
}
\label{fig:handmodel}
\end{figure}
