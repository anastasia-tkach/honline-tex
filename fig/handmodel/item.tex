\begin{figure}[t]
\centering
\begin{overpic} 
[width=\linewidth]
% [width=\linewidth,grid,tics=10]
{\currfiledir/item.pdf}
\myfigurename{}
\put(18,-3){Pose DOFs $\theta_{[i]}$}
\put(68,-3){Shape DOFs $\beta_{[i]}$}
\end{overpic}
\caption{
% 
The degrees of freedom of our optimization, where use a cartesian right-handed coordinate frame for translational DOFs. 
For pose parameters, global translation is represented by ${\theta_i | | i \in [0,1,2]}$ and rotation by ${\theta_i | | i \in [3,4,5]}$. We then color code DOFs according to whether they represent {\color[RGB]{53,120,109} flexion}, {\color[RGB]{212,144,133} twist}, and {\color[RGB]{172,72,100} abduction}. 
For shape parameters, we color code DOFs for
{\color[RGB]{172,72,100} lengths}, 
{\color[RGB]{212,144,133} radii}, 
{\color[RGB]{53,120,109} 3DOFs vertices (x,y,z)}, 
{\color[RGB]{129,190,163} 2DOFs vertices (x,y)}, and 
{\color[RGB]{120,120,120} passive DOFs (linearly dependent).}
% 
}
\label{fig:handmodel}
\end{figure}
