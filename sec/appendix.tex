\appendix
\section{Kalman Filter} 
\label{app:kalman}
Following the notation in~\cite{welch1995introduction}, given the measurements $z_n \in \mathbb{R}^M$, the Kalman Filter (KF) latent state $x_n \in \mathbb{R}^N$ update equations are:
% 
\begin{align}
x_n &= A x_{n - 1} +  w_{n - 1} \\
z_n &= J x_n + v_n
\end{align}
% 
where $w_n$ is a normally distributed process noise $p(w) \sim \mathcal{N}(0, Q)$, and $v_n$ is a normally distributed measurement noise $p(v) \sim \mathcal{N}(0, R)$. The matrix $A$ provides a linear estimate for state updates, while $J$ maps the state $x_n$ to the measurement $z_n$. As we are estimating a the shape of a single individual, we assume $A$ to be identity. Note that sensor noise is not normally distributed, thus our choice of $w$ can be considered an approximation \AT{not sure.. wouldn't sensor noise be accounted in \emph{measurement} noise?}:
% 
\begin{align}
x_n &= x_{n - 1} + w_{n - 1} \\
z_n &= J x_n + v_n
\end{align}
% 
In frame $n$, we provide an initial state estimate $\hat{x}_n^0$, while $\hat{x}_n$ is an improved estimate that accounts for the measurement $z_n$.
We define their corresponding covariances as follows, from which the update equations of \Table{kalman} can be derived:
% 
\begin{align}
P_n^0=\mathbb{E}[(x_n - \hat{x}_n^0)^T(x_n - \hat{x}_n^0)]\\
P_n =\mathbb{E}[(x_n - \hat{x}_n)^T(x_n - \hat{x}_n)]
\end{align}
% 
\begin{algorithm}
\begin{align*}
&\text{Time}                        &       &\text{Measurement} \\
\hat{x}_n^0 &= \hat{x}_{n - 1}      &       K_n &= P_n^0 J^T (J P_n^0 J^T + R)^{-1}\\
P_n^0 &= P_{n - 1} + Q              &       \hat{x}_n &= \hat{x}_n^0 + K_n (z_n - J \hat{x}_n^0) \\
&                                   &       P_n &= (I - K_n J) P_n^0
\end{align*}
\caption{
% 
% 
Kalman Filter update equations (with $A=I$).
% 
% 
}
\label{eq:kalman}
\label{tab:kalman}
\end{algorithm}

\subsection{Special case from the Section  \ref{sec:combining}}
Let us consider the special case when the measurement $z_n = x_n^*$ is in the same space at the estimated state $\hat{x}_n$, thus $J$ is equal to identity.
% 
\begin{align*}
\hat{x}_n 
&= \hat{x}_n^0 + P_n^0  (P_n^0 + R)^{-1}(z_n - \hat{x}_n^0) \\
&= (P_n^0 + R)(P_n^0 + R)^{-1}\hat{x}_n^0 + P_n^0  (P_n^0 + R)^{-1}(z_n - \hat{x}_n^0) \\
&= R(P_n^0 + R)^{-1}\hat{x}_n^0 + P_n^0  (P_n^0 + R)^{-1}z_n \\
% 
% 
P_n &= ((P_n^0 + R) (P_n^0 + R)^{-1} - P_n^0  (P_n^0 + R)^{-1}) P_n^0\\
&= R (P_n^0 + R)^{-1} P_n^0
\end{align*}
% 
Note how these two coincide with \Equation{combining} for product of two Gaussians  with $z_n = x_n^*$, $P_n^0 = \hat{\Sigma}_{n - 1}$ and $R = \Sigma_{n}$.

\section{Extended Kalman Filter}
The Extended Kalman Filter (EKF) estimates the latent state $x_n$ of a \emph{non-linear} system given the measurement $z_n$.
% 
\begin{align}
x_n = \tilde{F}(x_{n - 1},  w_{n - 1}) \\
z_n = F(x_n, v_n)
\end{align}
% 
% The function  that maps the  to the  applies shape and pose parameters to the hand model and computes the closest model points to sensor data points.
For hand tracking, the non-linearity is captured by the function $F(\cdot)$. The state $x_n$ captures parameters representing the pose and shape of our hand, and $F$ computes the correspondences between the measurement $z_n$ and 
our posed personalized model.
% 
\AT{arrived till here, will continue tomorrow (faculty interview now)}

The function $\tilde{F}(\cdot)$ relates the state at the previous time step to the state at current time step, in our case $\tilde{F}(\cdot)$ is an identity mapping. Thus, $\frac{ \partial \tilde{F}_{[i]}}{ \partial x_{[j]}}(\hat{x}_{n - 1}, 0) \equiv I$, 
$\frac{ \partial \tilde{F}_{[i]}}{ \partial w_{[j]}}(\hat{x}_{n - 1}, 0) \equiv I$ and $\frac{ \partial F_{[i]}}{ \partial v_{[j]}}(\hat{x}_n^0, 0) \equiv I$, where $I$ is an identity matrix of the corresponding size.
\begin{align}
x_n = x_{n - 1} + w_{n - 1} \\
z_n = F(x_n) + v_n 
\end{align}



%%%%%%%%%%%%%%%%%%%%%%%%%%%%%%%%%%%%%%%%%%%%
\begin{table}[t] 
\centering
\resizebox{0.42\textwidth}{!}{
\begin{tabular}{|c|c|}
\hline
Time Update & Measurement Update \\
\hline

$\hat{x}_n^0=\hat{x}_{n-1}$ &
$K_n = P_n^0 J_n^T(J_n P_n^0 J_n^T + R)^{-1}$ \\

$P_n^0 = P_{n-1} + Q$ &
$\hat{x}_n = \hat{x}_n^0 + K_n(z_n - F_n)$ \\

 &
$P_n = (I - K_n J_n)P_n^0$ \\

\hline
\end{tabular}
}
\caption{Extended Kalman Filter (with $\tilde{F}(\cdot)$ identity)}
\end{table}
%%%%%%%%%%%%%%%%%%%%%%%%%%%%%%%%%%%%%%%%%%%%

with $F_n = F(\hat{x}_n^0)$ and ${J_n}_{[i, j]} = \frac{ \partial F_{[i]}}{ \partial x_{[j]}}(\hat{x}_n^0)$.

\section{Iterated Extended Kalman Filter}
In our case the measurement function $F(\cdot)$ is nonlinear such that the Levenberg-Marquardt algorithm
$\hat{x}_n^{i + 1} = (J(\hat{x}_n^i) ^T J(\hat{x}_n^i) + \lambda I)^{-1} J(\hat{x}_n^i)^T F(\hat{x}_n^i)$ takes several iterations to converge. Thus we perform measurement update in several steps each time linearizing measurement function  $F(\cdot)$ around the updated value $\hat{x}_n^i$. We use the equations of Iterated Extended Kalman Filter (IEKF) presented in ~\cite{havlik2015performance}. The measurement update equations respectively for EKF and IEKF are presented in Table \ref{tab:ekf-iekf}. The time update equations remain the same as before.

\begin{table}[!h] 
\centering
\caption{EKF vs IEKF measurement update \label{tab:ekf-iekf}}
\resizebox{0.48\textwidth}{!}{
\begin{tabular}{|l|l|}
\hline
Extended Kalman Filter & Iterated Extended Kalman Filter \\
\hline

  &
for $i = 1...$ \\
  
$F_n = F(\hat{x}_n^0)$ &
$\:\:\: F_n^\textit{\color{accent}$i$} = F( \hat{x}_n^\textit{\color{accent}$i$})$ \\

${J_n}_{[u, v]} = \frac{ \partial F_{[u]}}{ \partial x_{[v]}}(\hat{x}_n^0)$ & 
$\:\:\: {J_n}_{[u, v]}^\textit{\color{accent}$i$} = \frac{ \partial F_{[u]}^\textit{\color{accent}$i$}}{ \partial x_{[v]}}(\hat{x}_n^\textit{\color{accent}$i$})$\\

$K_n = P_n^0 J_n^T(J_n P_n^0 J_n^T + R)^{-1}$ &
$\:\:\: K_n^\textit{\color{accent}$i$} = P_n^0 {J^\textit{\color{accent}$i$}_n}^T(J_n^\textit{\color{accent}$i$} P_n^0 {J^\textit{\color{accent}$i$}_n}^T + R)^{-1}$ \\

$\hat{x}_n = \hat{x}_n^0 + K_n(z_n - F_n)$ &
$\:\:\: \hat{x}_n^\textit{\color{accent}$i + 1$} = \hat{x}_n^0 + K_n^\textit{\color{accent}$i$}(z_n - F_n^\textit{\color{accent}$i$} - $\\

 &$\:\:\:\:\:\:\:\:\:\:\:\: - \textit{\color{accent} $J_n^i(\hat{x}_n^0 - \hat{x}_n^i)$})$ \\

 &
end \\
 
 & $\hat{x}_n = \hat{x}_n^\textit{\color{accent}$i$}$ \\
 
$P_n = (I - K_n J_n)P_n^0$ &
$P_n = (I - K_n^\textit{\color{accent}$i$} J_n^\textit{\color{accent}$i$})P_n^0$ \\

\hline
\end{tabular}
}
\end{table}

\section{Equivalence of IEIF and Optimization \ref{tab:iekf-like}} \label{app:ieif-lm}

To show that the optimizing the objective  \ref{tab:iekf-like} with LM algorithm is equivalent to IEKF, we first rewrite IEKF in information form as shown in \cite{anderson1979optimal}. The two algorithms presented in Table \ref{tab:ekf-eif} yield the same result.

\begin{table}[!h] 
\centering
\caption{EKF vs EIF measurement update \label{tab:ekf-eif}} 
\resizebox{0.48\textwidth}{!}{
\begin{tabular}{|l|l|}
\hline
Extended Kalman Filter & Extended Information Filter \\
\hline
 
 & $H_n = \left(P_n\right)^{-1}$ \\

$K_n = P_n^0 J_n^T(J_n P_n^0 J_n^T + R)^{-1}$ &
$H_n = H_n^0 + J_n^T R^{-1} J_n$  \\

$\hat{x}_n = \hat{x}_n^0 + K_n(z_n - F_n)$ &
$K_n = {H_n}^{-1} J_n^T R^{-1}$\\
 
$P_n = (I - K_n J_n)P_n^0$ &
$\hat{x}_n = \hat{x}_n^0 + K_n (z_n - F_n)$ \\

\hline
\end{tabular}
}
\end{table}

Table \ref{tab:ieif} presents the measurement update for Iterated Extended Information Filter, which is equivalent to Iterated Extended Kalman Filter. We assume that measurement noise is i.i.d. for each sensor sample, thus $R = \operatorname{diag}(r)$ with $r$ a positive scalar.

\begin{table}[!h] 
\centering
\caption{Iterated Extended Information Filter} \label{tab:ieif}
\begin{tabular}{|l|}
\hline

for $i = 1...$ \\

$\:\:\:\:\: H_n^\textit{\color{accent}$i$} = \frac{1}{r} (r H_n^0 +{J_n^\textit{\color{accent}$i$}}^T J_n^\textit{\color{accent}$i$})$  \\
  
$\:\:\:\:\: K_n^\textit{\color{accent}$i$} =\frac{r}{r} {H_n^\textit{\color{accent}$i$}}^{-1} {J_n^\textit{\color{accent}$i$}}^T$  \\

$\:\:\:\:\: \hat{x}_n^\textit{\color{accent}$i + 1$} = \hat{x}_n^0 + K_n^\textit{\color{accent}$i$}(z_n - F_n^\textit{\color{accent}$i$} - \textit{\color{accent} $J_n^i(\hat{x}_n^0 - \hat{x}_n^i)$})$ \\

end \\
 
$\hat{x}_n = \hat{x}_n^\textit{\color{accent}$i$}$ \\

\hline
\end{tabular}
\end{table}

 The LM update in Optimization \ref{tab:lm} has the following form: 
 
 \begin{equation}
 x_n^{i + 1}= x_n^{i} - (\bar{J}_n^T \bar{J}_n)^{-1} \bar{J}_n^T \bar{F}_n 
\end{equation}
 
 with $\bar{J}_n$ and $\bar{F}_n$ expanded below, $F_n^i = F(x_n^i)$, 
 $J_n^i = \frac{\partial F}{\partial x_n}(x_n^i)$, the measurement $z_n = d_n$ and $\hat{\Sigma}_n^{-1} = r H_{n - 1}$.
 
 \begin{align*}
x_n^{i + 1}= x_n^{i} - \left(
\left[
	\begin{array}{cc}
		- J_n^i \\
		\left(r H_{n - 1}\right)^{1/2} \\
	\end{array}
\right]^T 
\left[
	\begin{array}{c}
		- J_n^i \\
		\left(r H_{n - 1}\right)^{1/2} \\
	\end{array}
\right]
\right)^{-1} \times\\
\times \left[
	\begin{array}{cc}
		- J_n^i \\
		\left(r H_{n - 1}\right)^{1/2} \\
	\end{array}
\right]^T 
\left[
	\begin{array}{c}
		z_n - F_n^i \\
		\left(r H_{n - 1}\right)^{1/2}(x_n^i - \hat{x}_{n - 1}) \\
	\end{array}
\right] = \\
= x_n^{i} - \underbrace{\left({J_n^i}^T J_n^i + r H_{n - 1}\right)^{-1}}_{A} \times \\
\times \left( - {J_n^i}^T(z_n - F_n^i) + r H_{n - 1}(x_n^i - \hat{x}_{n - 1})\right)  = \\
= x_n^i +\underbrace{A {J_n^i}^T(z_n - F_n^i)}_B - A r H_{n - 1}(x_n^i - \hat{x}_{n - 1})  = \\
= \hat{x}_{n - 1} + B  - A r H_{n - 1}(x_n^i - \hat{x}_{n - 1}) + A A^{-1}( x_n^i - \hat{x}_{n - 1}) = \\
= \hat{x}_{n - 1} + B + A(- r H_{n - 1} + {J_n^i}^T J_n^i + r H_{n - 1})(x_n^i - \hat{x}_{n - 1}) = \\
= \hat{x}_{n - 1} + B + A {J_n^i}^T J_n^i (x_n^i - \hat{x}_{n - 1}) = \\
= \hat{x}_{n - 1} + \underbrace{\left({J_n^i}^T J_n^i + r H_{n - 1}\right)^{-1} {J_n^i}^T}_{K_n^i} \times \\
\times (z_n - F_n^i - J_n^i (\hat{x}_{n - 1} - x_n^i)),
 \end{align*}
 
 which is equivalent to the measurement update of IEIF from Table \ref{tab:ieif}, with $H_n^0 = H_{n-1}$ and $x_n^0 = \hat{x}_{n - 1}$.
 
