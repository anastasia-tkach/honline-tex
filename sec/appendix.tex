\appendix
\begin{algorithm}
\begin{align*}
&\text{Time}                        &       &\text{Measurement} \\
\hat{x}_n^0 &= \hat{x}_{n - 1}      &       K_n &= P_n^0 J^T (J P_n^0 J^T + R)^{-1}\\
P_n^0 &= P_{n - 1} + Q              &       \hat{x}_n &= \hat{x}_n^0 + K_n (z_n - J \hat{x}_n^0) \\
&                                   &       P_n &= (I - K_n J) P_n^0
\end{align*}
\caption{
% 
% 
Kalman Filter update equations (with $A=I$).
% 
% 
}
\label{eq:kalman}
\label{tab:kalman}
\end{algorithm}
\section{Overview on Kalman Filters}
\todo{Lorem ipsum dolor sit amet, consectetur adipisicing elit, sed do eiusmod tempor incididunt ut labore et dolore magna aliqua. Ut enim ad minim veniam, quis nostrud exercitation ullamco laboris nisi ut aliquip ex ea commodo consequat.}

\subsection{Kalman Filter (KF)} 
\label{app:kalman}
Following the notation in~\cite{welch1995introduction}, given the measurement $z_n \in \mathbb{R}^M$, the Kalman Filter (KF) latent state $x_n \in \mathbb{R}^N$ update equations are:
% 
\begin{align}
x_n &= A x_{n - 1} +  w_{n - 1} \\
z_n &= J x_n + v_n
\end{align}
% 
where $w_n$ is a normally distributed process noise $p(w) \sim \mathcal{N}(0, Q)$, and $v_n$ is a normally distributed measurement noise $p(v) \sim \mathcal{N}(0, R)$. The matrix $A$ provides a linear estimate for state updates, while $J$ maps the state $x_n$ to the measurement $z_n$.
We assume that while there are no new measurements, an estimate of the current hand parameters is given by the ones at the previous time-step up to Gaussian noise, that is, $x_n = x_{n-1} + w_{n-1}$. We also assume for now that depth sensor noise is normally distributed \AT{wouldn't sensor noise be accounted in \emph{measurement} noise?}, which actually is not the case. Thus, the state of our system can be approximated as:
% 
\begin{align}
x_n &= x_{n - 1} + w_{n - 1} \\
z_n &= J x_n + v_n
\end{align}
% 
In frame $n$, we provide an initial state estimate $\hat{x}_n^0$, while $\hat{x}_n$ is an improved estimate that accounts for the measurement $z_n$. We define their corresponding covariances as follows: 
% 
\begin{align}
P_n^0 &= \mathbb{E}[(x_n - \hat{x}_n^0)^T(x_n - \hat{x}_n^0)]\\
P_n   &= \mathbb{E}[(x_n - \hat{x}_n)^T(x_n - \hat{x}_n)]
\end{align}
%
Resulting in the time/measurement updates in \Table{kalman}.

\subsection{Extended Kalman Filter (EKF)}
\label{app:ekf}
The Extended Kalman Filter (EKF) estimates the latent state $x_n$ of a \emph{non-linear} system given measurements $z_n$.
% 
\begin{align}
x_n &= \tilde{F}(x_{n - 1},  w_{n - 1}) \\
z_n &= F(x_n, v_n)
\end{align}
% 
The function $F(\cdot)$ that maps the state $x_n$ to the measurement $z_n$ applies shape and pose parameters to the hand model and computes the closest model points to sensor data points. 
The function $\tilde{F}(\cdot)$ relates the state at the previous time step to the state at current time step, in our case $\tilde{F}(\cdot)$ is an identity mapping, therefore:
\begin{equation*}
\tfrac{ \partial \tilde{F}_{[i]}}{ \partial x_{[j]}}(\hat{x}_{n - 1}, 0) \equiv I
\quad
\tfrac{ \partial \tilde{F}_{[i]}}{ \partial w_{[j]}}(\hat{x}_{n - 1}, 0) \equiv I
\quad
\tfrac{ \partial F_{[i]}}{ \partial v_{[j]}}(\hat{x}_n^0, 0) \equiv I
\end{equation*}
% 
where $I$ is an identity matrix of the corresponding size.
% 
\AT{where is this chunk needed for? }
\AN{If these experssions are plugged in in the general IEFK equations, we get the below time and measurement update}
% 
\begin{align}
x_n &= x_{n - 1} + w_{n - 1} \\
z_n &= F(x_n) + v_n 
\end{align}
%
By defining $F_n = F(\hat{x}_n^0)$ and ${J_n}_{[i, j]} = \partial F_{[i]} / \partial x_{[j]}(\hat{x}_n^0)$, the EKF update equations can be written as reported in \Table{ekalman}.
%
\begin{algorithm}
\vspace{-.1in}
\begin{align*}
&\text{Time}                 &       &\text{Measurement} \\
\hat{x}_n^0 &=\hat{x}_{n-1}           &       K_n &= P_n^0 J_n^T(J_n P_n^0 J_n^T + R)^{-1}\\
P_n^0 &= P_{n-1} + Q                 &       \hat{x}_n &= \hat{x}_n^0 + K_n(z_n - F_n) \\
&                                   &       P_n &= (I - K_n J_n)P_n^0
\end{align*}
\caption{Extended Kalman Filter update equations (with linear $\tilde{F}$).}
\label{tab:ekalman}
\label{tab:extended}
\end{algorithm}

\subsection{Extended Information Filters}
\label{app:eif}
The EKF measurement updates can also be rewritten in the equivalent \emph{Extended Information Filter} form~\cite{optimal}; see \Table{eif}. Here, we assume that measurement noise is i.i.d. across sensor samples, therefore $R=rI$ where $r\in \mathbb{R}^+$. 
\begin{algorithm}[h]
\begin{gather*}
\begin{aligned}
&\text{Ex. Kalman Filter}                           &       &\text{Ex. Information Filter} \\
&                                                   &       H_n &= \left(P_n\right)^{-1} \\
K_n &= P_n^0 J_n^T(J_n P_n^0 J_n^T + R)^{-1}        &       H_n &= H_n^0 + J_n^T R^{-1} J_n \\
\hat{x}_n &= \hat{x}_n^0 + K_n(z_n - F_n)           &       K_n &= {H_n}^{-1} J_n^T R^{-1} \\
P_n &= (I - K_n J_n)P_n^0                           &       \hat{x}_n &= \hat{x}_n^0 + K_n (z_n - F_n)
\end{aligned}
\end{gather*}
\caption{
% 
% 
Mapping of an EKF into an EIF~\cite{optimal}.
% 
% 
}
\label{tab:eif}
\end{algorithm}

\subsection{Iterated Extended Kalman Filter (IEKF)}
\label{app:iekf}
As the measurement function $F(\cdot)$ from \Appendix{ekf} is nonlinear, the following LM optimization takes several iterations to converge: 
% 
\begin{equation}
\hat{x}_n^{i + 1} = (J(\hat{x}_n^i) ^T J(\hat{x}_n^i) + \lambda I)^{-1} J(\hat{x}_n^i)^T F(\hat{x}_n^i)   
\end{equation}
% 
To address this problem, we can perform measurement updates in several steps. In each step, we linearize the measurement function $F$ around the updated value $\hat{x}_n^i$, leading to the Iterated Extended Kalman Filter (IEKF) formulation~\cite{havlik2015performance}. The time update equation for IEKF is analogous to the one in \Table{extended}, while the measurement update is reported in \Table{iekf}. 
% 
\begin{algorithm}[b]
\begin{align*}
&\text{for} \quad i=1... i_{\text{max}} \\
& \quad \quad
\begin{aligned}
F_n^i &= F(\hat{x}_n^i) \Rightarrow {J_n^i}_{[u, v]} = \tfrac{\partial F^i_{[u]}}{ \partial x_{[v]}}(\hat{x}_n^i) \\
K_n^i &= P_n^0 {J^i_n}^T(J_n^i P_n^0 {J^i_n}^T + R)^{-1} \\
\hat{x}_n^{i+1} &= \hat{x}_n^0 + K_n^i(z_n - F_n^i -J_n^i(\hat{x}_n^0 - \hat{x}_n^i)) \\
\end{aligned} 
\\
& \text{end} \\
& \hat{x}_n = \hat{x}_n^i \\   
& P_n = (I - K_n^\textit{\color{accent}$i$} J_n^i) P_n^0
\end{align*}
\caption{Iterated EKF measurement update equations.}
\label{tab:iekf}
\end{algorithm}

\endinput % IGNORES WHAT FOLLOWS
%%%%%%%%%%%%%%%%%%%%%%%%%%%%%%%%%%%%%%%%%%%%%%%%%%%%%%%%%%%%%%%%%%%%%%%%
%%%%%%%%%%%%%%%%%%%%%%%%%%%%%%%%%%%%%%%%%%%%%%%%%%%%%%%%%%%%%%%%%%%%%%%%
%%%%%%%%%%%%%%%%%%%%%%%%%%%%%%%%%%%%%%%%%%%%%%%%%%%%%%%%%%%%%%%%%%%%%%%%
%%%%%%%%%%%%%%%%%%%%%%%%%%%%%%%%%%%%%%%%%%%%%%%%%%%%%%%%%%%%%%%%%%%%%%%%
%%%%%%%%%%%%%%%%%%%%%%%%%%%%%%%%%%%%%%%%%%%%%%%%%%%%%%%%%%%%%%%%%%%%%%%%
%%%%%%%%%%%%%%%%%%%%%%%%%%%%%%%%%%%%%%%%%%%%%%%%%%%%%%%%%%%%%%%%%%%%%%%%
%%%%%%%%%%%%%%%%%%%%%%%%%%%%%%%%%%%%%%%%%%%%%%%%%%%%%%%%%%%%%%%%%%%%%%%%

\begin{table}[h] 
\centering
\resizebox{0.48\textwidth}{!}{
\begin{tabular}{|l|l|}
\hline
Extended Kalman Filter & Iterated Extended Kalman Filter \\
\hline
&
for $i = 1...$ \\
  
$F_n = F(\hat{x}_n^0)$ &
$\:\:\: F_n^\textit{\color{accent}$i$} = F( \hat{x}_n^\textit{\color{accent}$i$})$ \\

${J_n}_{[u, v]} = \frac{ \partial F_{[u]}}{ \partial x_{[v]}}(\hat{x}_n^0)$ & 
$\:\:\: {J_n}_{[u, v]}^\textit{\color{accent}$i$} = \frac{ \partial F_{[u]}^\textit{\color{accent}$i$}}{ \partial x_{[v]}}(\hat{x}_n^\textit{\color{accent}$i$})$\\

$K_n = P_n^0 J_n^T(J_n P_n^0 J_n^T + R)^{-1}$ &
$\:\:\: K_n^\textit{\color{accent}$i$} = P_n^0 {J^\textit{\color{accent}$i$}_n}^T(J_n^\textit{\color{accent}$i$} P_n^0 {J^\textit{\color{accent}$i$}_n}^T + R)^{-1}$ \\

$\hat{x}_n = \hat{x}_n^0 + K_n(z_n - F_n)$ &
$\:\:\: \hat{x}_n^\textit{\color{accent}$i + 1$} = \hat{x}_n^0 + K_n^\textit{\color{accent}$i$}(z_n - F_n^\textit{\color{accent}$i$} - $\\

 &$\:\:\:\:\:\:\:\:\:\:\:\: - \textit{\color{accent} $J_n^i(\hat{x}_n^0 - \hat{x}_n^i)$})$ \\

 &
end \\
 
 & $\hat{x}_n = \hat{x}_n^\textit{\color{accent}$i$}$ \\
 
$P_n = (I - K_n J_n)P_n^0$ &
$P_n = (I - K_n^\textit{\color{accent}$i$} J_n^\textit{\color{accent}$i$})P_n^0$ \\

\hline
\end{tabular}
}
\caption{EKF vs. IEKF measurement update equations \label{tab:ekf-iekf}}
\end{table}

