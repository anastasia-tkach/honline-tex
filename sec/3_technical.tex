\section{Technical}
Given the $n$-th input depth frame $\mathcal{D}_n$ we first segment the hand point cloud~\cite{htrack}, and represent it in vectorized form as $d_n$. With $x_n = [\theta_n; \beta_n]$ we represent vector of coalesced pose and shape parameters at frame $n$. \TODO{Here describe the hand model, or perhaps that can be done in the intro?}

To achieve online \emph{joint} calibration and tracking, our algorithm presents two fundamental components which we refer to as \emph{intra-} and \emph{inter-frame} regression. The former gathers estimates from a single frame (see \Section{independent}), while the latter integrates this knowlege across frames (see \Section{combining}). We first demonstrate how inter-frame regression can be interpreted as a Kalman Filter (KF), highlight its shortcomings, and finally propose a joint \emph{inter/intra-frame} regression and interpret it as an Iterative Extended Kalman Filter (IEKF).

\begin{figure}[t]
\centering
\begin{overpic} 
[width=\linewidth]
% [width=\linewidth,grid,tics=10]
{\currfiledir item.pdf}
\myfigurename{}
\put(15,40.5){straight finger}
\put(65,40.5){bent finger}
% 
\put(28.5,9.5){\scriptsize $\beta_{[1]}$}
\put(18,19.5){\scriptsize $\beta_{[2]}$}
% 
\put(23,-3){\scriptsize $\beta_{[1]}^*$}
\put(40,-3){\scriptsize $\beta_{[1]}^+$}
\put(7.5,-3){\scriptsize $\beta_{[1]}^-$}
% 
\put(57,-3){\scriptsize $\beta_{[1]}^-$}
\put(74,-3){\scriptsize $\beta_{[1]}^*$}
\put(90,-3){\scriptsize $\beta_{[1]}^+$}
\end{overpic}
\caption{
% 
%
\textbf{(Intra-frame regression)} We abstract the hand shape/pose estimation problem from a single frame into the one of a simpler 2D stick-figure. Note, however, that this illustration is not hand-crafted, but is derived from numerical optimization executed on these simplified datasets. When the finger is straight (left), it is difficult to estimate the length of individual phalanges as the optimization problem is ill-posed. With a bent finger (right) the problem is better conditioned.
% 
% \AN{we need to tell that $E(\beta_1)$ is a registration energy} \AT{done}
We analyze the landscape of the registration energy {\color{anagreen}$E(\shape_{[1]})$}, and observe how estimation uncertainty relates to the width of the local minima valley. This uncertainty \AN{(posterior distribution of shape parameters after computing their estimate from the current frame)} can be estimated through a quadratic approximation {\color{anasalmon}$\tilde{E}(\shape_{[1]})$}, derived from the Jacobians of the registration energies.
% 
%
}
\label{fig:intra}
\end{figure}

\paragraph{Intra-frame regression}
Given a proper initialization $x_n^0$ and the input data $d_n$, intra-frame regression solves for a locally optimal $x_n^*$, as well as estimates the \emph{certainty/uncertainty} in the given solution. That is, this scheme provides a distribution for $x_n$ that is solely based on the knowledge in the depth frame $\mathcal{D}_n$. Estimating uncertainty in the solution is essential for template personalization. As we illustrate in \Figure{intra}, the confidence in regressed \emph{shape} parameters is conditional on the \emph{pose} of the current frame.
Rather than deriving this relationship from data, we follow the ideas in ~\todo{\cite{?}} and derive how this distribution can be approximated by a gaussian $\mathcal{N}(\star{x}_n, \star{\Sigma}_n)$. A fundamental observation is that the parameters of this distribution can be \emph{directly} obtained from the Jacobians of the tracking energies.

\begin{figure}[t]
\centering
\begin{overpic} 
[width=\linewidth]
% [width=\linewidth,grid,tics=10]
{\currfiledir/item.pdf}
\myfigurename{}
% 
\put(11,47){\scriptsize $\beta_2$}
\put(11,41){\scriptsize $\beta_1$}
% 
\put(3.5,40){\scriptsize \vertical{input data}}
\put(0,43){\scriptsize \vertical{$d_n$}}
%
\put(3.5,23){\scriptsize \vertical{intra-frame}}
\put(0,22){\scriptsize \vertical{$\mathcal{N}(\star{x}_n, \star{\Sigma}_n)$}}
% 
\put(3.5,4){\scriptsize \vertical{inter-frame}}
\put(0,3){\scriptsize \vertical{$\mathcal{N}(\hat{x}_n, \hat{\Sigma}_n)$}}
% 
\put(11,-2){\small $n=1$}
\put(30,-2){\small $n=2$}
\put(49,-2){\small $n=3$}
\put(68,-2){\small $n=4$}
\put(86.5,-2){\small $n=5$}
% 
\end{overpic}
\caption{
% 
\textbf{(Inter-frame regression)} We visualize several temporally sorted frames of input data $d_n$, the uncertainty ellipsoid $\star{\Sigma}_n$ estimated by intra-frame regression, and the \emph{online} uncertainty estimate $\hat{\Sigma}_n$. For illustration purposes, we only display the 2-dimensional ellipsoids representing the covariance of $\beta_1$ and $\beta_2$. Although $\star{\Sigma}_1=\star{\Sigma}_5$, observe how $\hat{\Sigma}_1 \succ \hat{\Sigma}_5$, that is in the last frame we have a confident guess as the information from frames $2:4$ has been embedded in our calibrated model; see also \Figure{merging}.
%  
}
\label{fig:inter}
\end{figure}

\paragraph{Inter-frame regression}
Unlike pose parameters, shape parameters are \emph{persistent} over time: we observe the same user performing in front of the camera. Further, sufficient information to estimate certain shape parameters is simply not available in certain frames. For example, by observing a straight finger like the one in \Figure{intra}, it is difficult to estimate the length of a phalanx, therefore knowledge must be gathered from a \emph{collection} of frames capturing the user in different poses. Rather than collecting a persistent model from a set of pre-recorded calibration data as in~\cite{taylor_sig16}, we approach the problem from an \emph{online-modeling} point of view. In what follows, we detail a principled method to temporally integrate parameter estimates in a way that also accounts for their confidence.

\subsection{Intra-frame regression -- $\mathcal{N}(\star{x}_n, \star{\Sigma}_n)$}
\label{sec:independent}
\label{sec:intra}
% 
Given $d_n$ the point cloud at frame $n$, and an initialization of hand pose from the previous frame $x_n^0 = x_{n - 1}^*$ we find an independent estimate of $x_n^*$ by local iterative optimization (a-la Levenberg) of the multi-objective energy function:
% 
\begin{equation}
x_n^* = \argmin_{x_n} \sum_{\tau \in \mathcal{T}} E_{\tau}(d_n, x_n) 
\label{eq:energies}
\end{equation}
% 
Where the terms $\mathcal{T}$ ensure that:
%
% \vspace{-.5\parskip}
\begin{equation*}
\setlength{\jot}{0pt}
\begin{aligned}
\text{\textbf{d2m}} & \quad \text{data points are explained by the model} \\ 
\text{\textbf{m2d}} & \quad \text{the model lies in the sensor visual-hull} \\
\text{\textbf{shape-prior}} & \quad \text{the model shape should be plausible} \\
\text{\textbf{pose-prior}} & \quad \text{the model pose should be plausible} \\
\text{\textbf{smooth}} & \quad \text{the recorded sequence is smooth} \\
\text{\textbf{collision}} & \quad \text{fingers do not interpenetrate} \\
\text{\textbf{limits}} & \quad \text{joint limits are satisfied} 
\end{aligned}
\end{equation*}
The energy terms in the objective function are detailed in~\cite{tkach2016sphere} and~\cite{htrack}, with the exception of \emph{shape-prior} that is discussed in \Section{shapeprior}.

\begin{algorithm}
\begin{equation*}
\begin{aligned}
>> \hat{x}_n &= \argmax_{x_n} \: \underbrace{\log \left( P(x_n^*|x_n) \: P(x_n |\hat{x}_{n - 1}) \right)}_{L(x_n)}    
\\
\end{aligned}
\end{equation*}

\begin{equation*}
\begin{aligned}
>> P(x_n |x_n^*) &= \exp \left( -\tfrac{1}{2}(x_n^* - x_n )^T {\star{\Sigma}_n}^{-1}(x_n^* - x_n )\right)
% \mathcal{N}(x_n^*,\todo{\Sigma_n})
\\
P(x_n |\hat{x}_{n - 1}) &= 
% \mathcal{N}(\hat{x}_{n - 1}, \hat{\Sigma}_{n-1})
\exp \left( -\tfrac{1}{2} (x_n - \hat{x}_{n - 1} )^T \hat{\Sigma}_{n - 1}^{-1} (x_n - \hat{x}_{n - 1} )\right)
\\
\hat{\Sigma}_n^{-1} = 
\left.\tfrac{\partial^2 L}{\partial x_n^2}\right|_{\hat{x}_n} &\approx 
\left[
    \begin{array}{cc}
        \sqrt{{\star{\Sigma}_n}^{-1}} \\
        \sqrt{\hat{\Sigma}_{n - 1}^{-1}} \\
    \end{array}
\right]^T
\left[
    \begin{array}{c}
        \sqrt{{\star{\Sigma}_n}^{-1}} \\
        \sqrt{\hat{\Sigma}_{n - 1}^{-1}} \\
    \end{array}
\right]
= {\star{\Sigma}_n}^{-1} + \hat{\Sigma}_{n-1}^{-1}
\end{aligned}
\end{equation*}
\caption{
% 
% 
\emph{Split} inter-frame regression -- Kalman Filter (KF)
% 
% 
}
\label{tab:interframe}
\label{tab:kf-like}
\end{algorithm}
\paragraph{Laplace approximation}
% \subsection{Posterior distribution of parameters after taking a measurement into account}
\label{sec:posterior}
While the optimization problem in \Equation{energies} estimates pose/shape parameters, information regarding the confidence of the estimate is not directly available. To retrieve this information, we first convert the \Equation{energies} in a probabilistic format, and from this formulation we then derive our uncertainties. 
% 
Towards this goal, we rewrite the \emph{d2m} and \emph{m2d} terms in \Eq{energies} as:
% 
\begin{align}
E_{\tau}(d_n, x_n) = \|I_{\tau} d_n - F_{\tau} (x_n)\|_2^2,
\end{align} 
where $I_\tau$ are identity matrices, and $I_{m2d}(i,i)=0$ if the rasterized 2D template pixel $\mathbf{p}_i$ lies within the sensor silhouette image; see \cite{htrack}. We also concatenate the two terms in pair of column vectors:
% 
\begin{align*}
F(x_n) = \left[F_{d2m}(x_n); F_{m2d}(x_n)\right] 
\:\:\text{and}\:\: 
d_n = \left[(I_{d2m} d_n); (I_{m2d} d_n) \right],
\end{align*}
% 
leading to a compact posterior distribution representation:
% 
\begin{align}
P(d_n|x_n) &= \exp \left( - \tfrac{1}{2}(d_n - F(x_n))^T (d_n - F(x_n)) \right)
\label{eq:posterior}
\end{align}
% 
\Equation{energies} is then rewritten in probabilistic form, where we temporarily omit the frame index $n$ for conveniency:
%
\begin{align}
x^* &= \argmin_{x} \underbrace{\log  P(d|x)}_{L(x)}
\label{eq:independent}
\end{align}
%
We now perform a second-order Taylor expansion of the log-likelihood of the data $L(x)$ around the \emph{optimal} solution $x^*$:
%
\begin{align}
L(x) \approx \tilde{L}(x) = L(x^*)   
+ \tfrac{\partial L(x^*) }{\partial x}  \Delta x 
+ \tfrac{1}{2} \Delta x^T\tfrac{\partial^2 L(x^*)}{{\partial x}^2} \Delta x + \text{h.o.t.}
\label{eq:taylor}
\end{align}
%
where $\Delta x=x - x^*$, and with an abuse of notation, as $\partial f(\star{x})/\partial x$ we indicate the partial derivative of $f(x)$ evaluated at $\star{x}$. Note how the Jacobian and the Hessian are respectively zero and positive definite at our optimal point $x^*$:
%
\begin{align}
\tfrac{\partial L(x^*)}{\partial x} &= - 2 F(x^*)^T 
\tfrac{\partial F(x^*)}{\partial x} = 0 
\label{eq:taylor-jacobian}
\\
\tfrac{\partial^2 L(x^*)}{\partial x^2} 
% = 2 \left.\tfrac{\partial F(x)}{\partial x}\right|_{x^*}^T \left.\tfrac{\partial F(x)}{\partial x}\right|_{x^*} + \\ 2 F(x^*)^T  \left. \tfrac{\partial \partial F (x^*) }{\partial \partial x} \right|_{x^*}
& \approx 2 \tfrac{\partial F(x^*)}{\partial x}^T \tfrac{\partial F(x^*)}{\partial x}
= {\star{\Sigma}}^{-1} \succ 0
\label{eq:taylor-hessian}
\end{align}
% 
From \Equation{taylor}, remembering that $\tilde P(d|x) = \exp (\tilde{L}(x))$, we can then derive the \emph{approximated} posterior distribution:
%
\begin{align}
\tilde{P}(d|x) = \exp\left(- \tfrac{1}{2}(x - x^*)^T {\star{\Sigma}}^{-1}  (x - x^*) \right) = \mathcal{N}\left(\star{x}, \star{\Sigma} \right)
\end{align}
%
That is, after processing the information in a frame $d_n$, the sought-after quadratic approximation of posterior distribution of model parameters is a \emph{normal} distribution $\mathcal{N}\left(x_n^*, \star\Sigma_n \right)$.
Its mean is the solution of the (iterative) optimization problem in \Equation{energies}; its covariance is computed according to \Equation{taylor-hessian}, that is, from the Jacobians of the energy terms $\{E_{\text{d2m}}, E_{\text{m2d}}\}$ at the optimal solution.

\subsection{Inter-frame regression -- $\mathcal{N}(\hat{x}_n, \hat{\Sigma}_n) | \mathcal{N}(\star{x}_n, \star{\Sigma}_n)$ } 
% \subsection{Merging independent measurements}
\label{sec:combining}
\label{sec:inter}
Given the probabilistic interpretation in \Section{independent}, a temporal regression over parameters can be obtained by cumulating the posterior distributions over the set of previous frames. This leads to the to the following pair of inductive update equations:
% 
\begin{align}
\mathcal{N}(\hat{x}_1, \hat{\Sigma}_1) &= \mathcal{N}(\star{x}_1, \star{\Sigma}_1) \\
\mathcal{N}(\hat{x}_n, \hat{\Sigma}_n) &= \mathcal{N}(\hat{x}_{n-1}, \hat{\Sigma}_{n-1}) \mathcal{N}(\star{x}_n, \star{\Sigma}_n)
\end{align}
% 
By applying the product of Gaussians rule ~\cite{petersen2008matrix}:
% 
\begin{align}
\begin{split}
\hat{x}_{n} &= \star\Sigma_{n} (\hat{\Sigma}_{n-1} + \star\Sigma_{n})^{-1} \hat{x}_{n-1} + 
\hat{\Sigma}_{n-1} (\hat{\Sigma}_{n-1} + \star\Sigma_n)^{-1} x_n^*
\\
\hat{\Sigma}_n &= \hat{\Sigma}_{n-1} (\hat{\Sigma}_{n-1} + {\star\Sigma_n})^{-1} \star\Sigma_n
\label{eq:combining}
\end{split}
\end{align}
% 
In \Appendix{kalman}, we shown how \Equation{combining} is equivalent to the Kalman Filter (KF) update equations in \Table{interframe}, with measurement $x_n^*$, and measurement noise covariance $\star\Sigma_n$. This observation is fundamental as, under the same assumptions we made in \Appendix{kalman}, \cite[\todo{Pg.?}]{maybeck1979stochastic} noted how a KF is provably \emph{optimal}.
This optimization will be used as a \emph{baseline}, as it is arguably the simplest way of achieving an online parameter regression: by treating the results of the independent solve $\mathcal{N}(\star{x}_n, \star{\Sigma}_n)$ as the measurements in a KF.

\begin{algorithm}
\begin{equation*}
\begin{aligned}
\hat{x}_n &= \argmax_{x_n} \:
\underbrace{\log \left( P(d_n|x_n) \: P(x_n |\hat{x}_{n - 1}) \right)}_{L(x_n)}   
\\
\end{aligned}
\end{equation*}

\begin{equation*}
\begin{aligned}
P(d_n|x_n) &= \exp \left( -\tfrac{1}{2} (d_n - F(x_n))^T (d_n - F(x_n)) \right)
\\
P(x_n |\hat{x}_{n - 1}) &= \exp \left( -\tfrac{1}{2}(x_n - \hat{x}_{n - 1})^T \hat{\Sigma}_{n - 1}^{-1} (x_n - \hat{x}_{n - 1})\right)
\\
\hat{\Sigma}_n^{-1} = 
\left.\tfrac{\partial^2 L}{\partial x_n^2}\right|_{\hat{x}_n} &\approx 
\left[
	\begin{array}{cc}
		- \tfrac{\partial F(\hat{x}_n)} {\partial x_n} \\
		\sqrt{\hat{\Sigma}_{n - 1}^{-1}} \\
	\end{array}
\right]^T 
\left[
	\begin{array}{c}
		\tfrac{\partial F(\hat{x}_n)} {\partial x_n} \\
		\sqrt{\hat{\Sigma}_{n - 1}^{-1}} \\
	\end{array}
\right] 
= \Sigma_{n}^{-1} + \hat{\Sigma}_{n-1}^{-1}  
\end{aligned}
\end{equation*}
\caption{\emph{Joint} inter-frame regression -- Iterated Extended KF (IEKF)}
\label{alg:interframeplus}
\label{alg:iekf-like}
\label{tab:iekf-like}
\end{algorithm}

%%%%%%%%%%%%%%%%%%%%%%%%%%%%%%%%%%%%%%%%%%%%
% \begin{table}[t]
% \centering
% \resizebox{0.48\textwidth}{!}{
% \begin{tabular}{|c|}
% \hline
% $\hat{x}_n = \operatorname{argmax}_{x_n}
% \underbrace{\log \left( P(d_n|x_n) P(x_n |\hat{x}_{n - 1}) \right)}_{L(x_n)}$\\
% with \\
% $P(d_n|x_n) = \exp \left( \tfrac{1}{2} - (d_n - F(x_n))^T (d_n - F(x_n)) \right)$ \\
% $P(x_n |\hat{x}_{n - 1}) = \exp \left( \tfrac{1}{2} -(x_n - \hat{x}_{n - 1})^T \hat{\Sigma}_{n - 1}^{-1} (x_n - \hat{x}_{n - 1})\right)$ \\
% \\
% $\hat{\Sigma}_n^{-1} =  \left. \tfrac{\partial \partial L(x_n)}{\partial x_n \partial x_n}\right|_{\hat{x}_n} \approx $ \\
% \\
% $\left[
%     \begin{array}{cc}
%         - \left. \tfrac{\partial F(x_n)} {\partial x_n} \right|_{\hat{x}_n} \\
%         \left(\hat{\Sigma}_{n - 1}^{-1}\right)^{1/2} \\
%     \end{array}
% \right]^T
% \left[
%     \begin{array}{c}
%         -\left. \tfrac{\partial F(x_n)} {\partial x_n} \right|_{\hat{x}_n} \\
%         \left(\hat{\Sigma}_{n - 1}^{-1}\right)^{1/2} \\
%     \end{array}
% \right] = $\\
% \\
% $ = \hat{\Sigma}_{n-1}^{-1} + \left. \tfrac{\partial F(x_n)} {\partial x_n} \right|_{\hat{x}_n}^T
% \left. \tfrac{\partial F(x_n)} {\partial x_n} \right|_{\hat{x}_n} = \hat{\Sigma}_{n-1}^{-1}  + \Sigma_{n}^{-1}$    \\
% \hline
% \end{tabular}
% }
% \end{table}
\subsection{Joint inter/intra frame regression}
% Note that the best-fitting parameters $x_n^*$ for the given the single input frame $d_n$ are computed using local LM optimization. Thus, it is crucial that optimization starts in a sensible region of parameters space.
The optimization in \Tab{interframe} does not provide any information about the current estimate of the parameters $\hat{x}_n$ to the independent solve described in \Section{independent}. This could be problematic, as in this case \Eq{energies} does not leverage any temporal information aside from initialization, while relying on a sufficiently good initialization to compute $\mathcal{N}(\star{x}_n, \star{\Sigma}_n)$. One potential way to increase the robustness of the algorithm would be to include this information in \Eq{independent}:
% 
\begin{align}
x_n^* &= \argmin_{x_n} \log  P(d_n|x_n) P(x_n |\hat{x}_{n - 1}) 
\label{eq:independenttemp}
\end{align}
% 
Instead, we propose to coalesce the inter and intra frame optimization resulting in the \emph{joint} intra/inter online regression scheme in \Table{iekf-like}. Interestingly, we demonstrate that optimizing the objective in~\Table{iekf-like} with a Levemberg-Marquardt optimization is equivalent to an Iterated Extended Kalman Filter (IEKF), with \todo{measurement update... what?}.
% 
Notice how this optimization, representing our online tracking/modeling system, jointly considers the measurement $d_n$ as well as past estimates $\mathcal{N}(\hat{x}_n, \hat{\Sigma}_n)$.
\newpage

\subsection{Shape prior (TODO)}
\label{sec:shapeprior}
\TODO{Anastasia?}
\AN{The shape prior should only be used in the first frame, because afterwards it will be incorporated inside of Kalman prior}. \AT{As this is quite advanced, this can be discussed at the very end of the section}
