\section{Technical}
Given the $n$-th input depth frame $\mathcal{D}_n$ we first segment the hand point cloud~\cite{htrack}, and represent it in vectorized form as $d_n$. With $x_n = [\theta_n; \beta_n]$ we represent vector of coalesced pose and shape parameters at frame $n$. \TODO{Here describe the hand model, or perhaps that can be done in the intro?}

To achieve online \emph{joint} calibration and tracking, our algorithm presents two fundamental components which we refer to as \emph{intra-} and \emph{inter-frame} regression. The former gathers estimates from a single frame (see \Section{independent}), while the latter integrates this knowlege across frames (see \Section{combining}). We first demonstrate how inter-frame regression can be interpreted as a Kalman Filter (KF), highlight its shortcomings, and finally propose a joint \emph{inter/intra-frame} regression and interpret it as an Iterative Extended Kalman Filter (IEKF).

\begin{figure}[t]
\centering
\begin{overpic} 
[width=\linewidth]
% [width=\linewidth,grid,tics=10]
{\currfiledir item.pdf}
\myfigurename{}
\put(15,40.5){straight finger}
\put(65,40.5){bent finger}
% 
\put(23,-3){$x_s^*$}
\put(40,-3){$x_s^+$}
\put(7.5,-3){$x_s^-$}
% 
\put(57,-3){$x_b^-$}
\put(74,-3){$x_b^*$}
\put(90,-3){$x_b^+$}
\end{overpic}
\caption{
% 
% 
\textbf{(Intra-frame regression)} We abstract the hand shape/pose estimation problem from a single frame into the one of a simpler 2D stick-figure.  When the finger is straight (left), it is difficult to estimate finger length as the optimization problem is ill-posed, while with a bent finger (right) the problem is better conditioned.
% 
We analyze the landscape of the registration energies {\color{anagreen}$E(\shape_1)$}, and observe how estimation uncertainty relates to the width of the local minima valley. This certainty can be estimated through a quadratic approximation {\color{anasalmon}$\tilde{E}(\shape_1)$}, derived from the Jacobians of the registration energies.
% 
%
}
\label{fig:intra}
\end{figure}

\paragraph{Intra-frame regression}
Given a proper initialization $x_n^0$ and the input data $d_n$, intra-frame regression solves for a locally optimal $x_n^*$, as well as estimates the \emph{uncertainty} in the given solution. That is, this scheme provides a distribution for $x_n$ that is solely based on the knowledge in the depth frame $\mathcal{D}_n$. Estimating uncertainty in the solution is essential for shape estimation. As we illustrate in \Figure{intra}, the confidence in regressed \emph{shape} parameters is conditional on the \emph{pose} of the current frame.
Rather than deriving this relationship from data, we follow the ideas in ~\todo{\cite{?}} and derive how this distribution can be approximated by a gaussian $\mathcal{N}(\star{x}_n, \star{\Sigma}_n)$. A fundamental observation is that the parameters of this distribution can be \emph{directly} obtained from the Jacobians of the tracking energies.

\begin{figure}[t]
\centering
\begin{overpic} 
[width=\linewidth]
% [width=\linewidth,grid,tics=10]
{\currfiledir/item.pdf}
\myfigurename{}
% 
\put(18.5,41){\scriptsize $\beta_{[1]}$}
\put(56,47){\scriptsize $\beta_{[2]}$}
\put(94,50){\scriptsize $\beta_{[3]}$}
% 
\put(75,46){\scriptsize $\theta_{[2]}$}
\put(37.5,41){\scriptsize $\theta_{[1]}$}
% 
\put(3.5,40){\scriptsize \vertical{input data}}
\put(0,43){\scriptsize \vertical{$d_n$}}
%
\put(3.5,23){\scriptsize \vertical{intra-frame}}
\put(0,22){\scriptsize \vertical{$\mathcal{N}(\star{x}_n, \star{\Sigma}_n)$}}
% 
\put(3.5,4){\scriptsize \vertical{inter-frame}}
\put(0,3){\scriptsize \vertical{$\mathcal{N}(\hat{x}_n, \hat{\Sigma}_n)$}}
% 
\put(12.0,-2){\scriptsize $n=1$}
\put(30.5,-2){\scriptsize $n=2$}
\put(49.5,-2){\scriptsize $n=3$}
\put(68.5,-2){\scriptsize $n=4$}
\put(87.0,-2){\scriptsize $n=5$}
% 
\end{overpic}
\caption{
% 
\textbf{(Inter-frame regression)} We visualize several temporally sorted frames of input data $d_n$, the uncertainty ellipsoid $\star{\Sigma}_n$ estimated by intra-frame regression, and the \emph{online} uncertainty estimate $\hat{\Sigma}_n$. For illustration purposes, we only display the two-dimensional ellipsoids representing the covariance of $\beta_{[1]}$ and $\beta_{[2]}$. Although $\star{\Sigma}_1=\star{\Sigma}_5$, observe how $\hat{\Sigma}_1 \succ \hat{\Sigma}_5$: in the last frame we have a confident estimate as the information from frames {\scriptsize $2:4$} has been integrated.
% 
Further, notice how, even though the parameter $\beta_{[2]}$ was not observed directly in any of the presented frames, its value was inferred from the measurements \AN{high certainty measurements. It is important to say that they are high certainty}of~$(\beta_{[1]})_{n=2}$~and~$(\beta_{[1]} + \beta_{[2]})_{n=3}$.
%; see also \Figure{merging}.
%  
}
\label{fig:inter}
\end{figure}

\paragraph{Inter-frame regression}
Differently from pose parameters shape parameters are \emph{persistent} over time: we observe the same user performing in front of the camera. Moreover, in certain frames sufficient information to estimate certain shape parameters is simply not available. For example, by observing a straight finger like the one in \Figure{intra}, it is difficult to estimate the length of a phalanx, therefore knowledge must be gathered from a \emph{collection} of frames. Rather than collecting a persistent model from a set of pre-recorded calibration data as in~\cite{taylor_sig16}, we approach the problem from a online-modeling point of view. In what follows, we detail a principled method to \emph{temporally integrate} parameter estimates in a way that also accounts for their confidence.

\subsection{Intra-frame regression -- $\mathcal{N}(\star{x}_n, \star{\Sigma}_n)$}
\label{sec:independent}
\label{sec:intra}
% 
Given $d_n$ the point cloud at frame $n$, and an initialization of hand pose from the previous frame $x_n^0 = x_{n - 1}^*$ we find an independent estimate of $x_n^*$ by local iterative optimization (a-la Levenberg) of the multi-objective energy function:
% 
\begin{equation}
x_n^* = \argmin_{x_n} \sum_{\tau \in \mathcal{T}} E_{\tau}(d_n, x_n) 
\label{eq:energies}
\end{equation}
% 
Where the terms $\mathcal{T}$ ensure that:
%
% \vspace{-.5\parskip}
\begin{equation*}
\setlength{\jot}{0pt}
\begin{aligned}
\text{\textbf{d2m}} & \quad \text{data points are explained by the model} \\ 
\text{\textbf{m2d}} & \quad \text{the model lies in the sensor visual-hull} \\
\text{\textbf{shape-prior}} & \quad \text{the model shape should be plausible} \\
\text{\textbf{pose-prior}} & \quad \text{the model pose should be plausible} \\
\text{\textbf{collision}} & \quad \text{fingers do not interpenetrate} \\
\text{\textbf{limits}} & \quad \text{joint limits are satisfied} \\
\text{\textbf{smooth}} & \quad \text{\todo{ought to include -> avoid jitter}}
\end{aligned}
\end{equation*}
The energy terms in the objective function are detailed in~\cite{tkach2016sphere} and~\cite{htrack}, with the exception of \emph{shape-prior} that is discussed in \Section{shapeprior}.

\begin{algorithm}
\begin{equation*}
\begin{aligned}
x_n^* &= \argmax_{x_n} \: \log P(d_n|x_n)  \quad \text{\TODO{remove?}}
\\
\hat{x}_n &= \argmax_{x_n} \: \underbrace{\log \left( P(x_n|x_n^*) \: P(x_n |\hat{x}_{n - 1}) \right)}_{L(x_n)}    
\\
\end{aligned}
\end{equation*}

\begin{equation*}
\begin{aligned}
P(x_n |x_n^*) &= \exp \left( - (x_n - x_n^* )^T \todo{\Sigma}_n^{-1}(x_n - x_n^* )\right)
% \mathcal{N}(x_n^*,\todo{\Sigma_n})
\\
P(x_n |\hat{x}_{n - 1}) &= 
% \mathcal{N}(\hat{x}_{n - 1}, \hat{\Sigma}_{n-1})
\exp \left( - (x_n - \hat{x}_{n - 1} )^T \hat{\Sigma}_{n - 1}^{-1} (x_n - \hat{x}_{n - 1} )\right)
\\
\hat{\Sigma}_n^{-1} = 
\left.\tfrac{\partial^2 L}{\partial x_n^2}\right|_{\hat{x}_n} &\approx 
\todo{
\left[
    \begin{array}{cc}
        \sqrt{\Sigma_n^{-1}} \\
        \sqrt{\hat{\Sigma}_{n - 1}^{-1}} \\
    \end{array}
\right]^T
\left[
    \begin{array}{c}
        \sqrt{\Sigma_n^{-1}} \\
        \sqrt{\hat{\Sigma}_{n - 1}^{-1}} \\
    \end{array}
\right]
} %< accent
=
\hat{\Sigma}_{n-1}^{-1}  + \todo{\Sigma_{n}}^{-1}
\end{aligned}
\end{equation*}
\caption{
% 
% 
Inter-frame regression formulated as a probabilistic KF. 
\AT{what's the point of the big arrays? They tell me nothing as a reader?} 
% 
% 
}
\label{alg:interframe}
\label{alg:kf-like} %< newish label
\label{tab:kf-like} %< old label
\end{algorithm}
\paragraph{Laplace approximation}
% \subsection{Posterior distribution of parameters after taking a measurement into account}
\label{sec:posterior}
While the optimization problem in \Equation{energies} estimates pose/shape parameters, information regarding the confidence of the estimate is not directly available. To retrieve this information, we first convert the \Equation{energies} in a probabilistic format, and from this formulation we then derive our uncertainties. Towards this goal, we first rewrite the energy terms in \Eq{energies} as:
% 
\begin{align}
E_{\tau}(d_n, x_n) = \|I_{\tau} d_n - F_{\tau} (x_n)\|_2^2,
\end{align} 
where $I_{d2m}$ and $I_{m2d}$ are identity matrices, and $I_{m2d}(i,i)=0$ if the rasterized 2D template pixel $\mathbf{p}_i$ lies within the sensor silhouette image \AT{correct?} (see \cite{htrack}), and the remaining $I_{\tau}$ are zero matrices.
% energy terms  can be rewritten in a form $$ with ,  a permutation(/reduction?) matrix that removes from $d_n$ the \todo{entries that are not the closest for any model point} \AT{meaning?},
For brevity of notation, we also denote:
% 
\begin{align*}
F(x_n) &= \left[F_{d2m}^T(x_n), \quad ..., \quad F_{pose}^T(x_n)\right]^T \\
d_n &= \left[(I_{d2m} d_n)^T, \quad ..., \quad (I_{pose} d_n)^T\right]^T
\end{align*}
% 
leading to a compact posterior distribution representation:
% 
\begin{align}
P(d_n|x_n) &= \exp \left( - \tfrac{1}{2}(d_n - F(x_n))^T (d_n - F(x_n)) \right)
\label{eq:posterior}
\end{align}
% 
\Equation{energies} is then rewritten in probabilistic form, where we temporarily omit the frame index $n$ for conveniency:
%
\begin{align}
x^* &= \argmin_{x} \underbrace{\log  P(d|x)}_{L(x)}
\label{eq:independent}
\end{align}
%
We now perform a second-order Taylor expansion of the log-likelihood of the data $L(x)$ around the \emph{optimal} solution $x^*$:
%
\begin{align*}
L(x) \approx \tilde{L}(x) = L(x^*)   
+ \tfrac{\partial L(x^*) }{\partial x}  \Delta x 
+ \tfrac{1}{2} \Delta x^T\tfrac{\partial^2 L(x^*)}{{\partial x}^2} \Delta x + \text{h.o.t.}
\end{align*}
%
where $\Delta x=x - x^*$, and with an abuse of notation, as $\partial f(x_0)/\partial x$ we indicate the partial derivative of $f(x)$ evaluated at the point $x_0$. Note how the Jacobian and the Hessian are respectively zero and positive definite at our critical point $x^*$:
%
\begin{align}
\tfrac{\partial L(x^*)}{\partial x} &= - 2 F(x^*)^T 
\tfrac{\partial F(x^*)}{\partial x} = 0 
\label{eq:taylor-jacobian}
\\
\todo{\Sigma} = \tfrac{\partial^2 L(x^*)}{\partial x^2} 
% = 2 \left.\tfrac{\partial F(x)}{\partial x}\right|_{x^*}^T \left.\tfrac{\partial F(x)}{\partial x}\right|_{x^*} + \\ 2 F(x^*)^T  \left. \tfrac{\partial \partial F (x^*) }{\partial \partial x} \right|_{x^*}
& \approx 2
\tfrac{\partial F(x^*)}{\partial x}^T  
\tfrac{\partial F(x^*)}{\partial x}
\succ 0
\label{eq:taylor-hessian}
\end{align}
% 
Remembering that $\tilde P(d|x) = \exp (\tilde{L}(x))$, we can then derive the \emph{approximated} posterior distribution:
%
\begin{align}
\tilde{P}(d|x) = \exp\left(- \tfrac{1}{2}(x - x^*)^T \todo{\Sigma}^{-1}  (x - x^*) \right) = \mathcal{N}\left(\star{x}, \todo{\Sigma} \right)
\end{align}
%
\AT{can we rename $\Sigma$ into $\star{\Sigma}$ for clarity of notation?}
That is, after processing the information in a frame $d_n$, the desired quadratic approximation of posterior distribution of model parameters is a \emph{normal} distribution $\mathcal{N}\left(x_n^*, \Sigma_n \right)$.
Its mean is the (final) solution of the (iterative) optimization problem in \Equation{energies}; its variance is computed according to \Equation{taylor-jacobian}, that is, from the Jacobian computed in the last iteration of~\Equation{energies}. 
% 
\AT{ITERATIVE?}
% \AT{we might have to also write the iterative version of \Equation{energies}; it sounds a bit... hand-wavy, ahahehehjaja}.

\subsection{Inter-frame regression -- $\mathcal{N}(\hat{x}_n, \hat{\Sigma}_n) | \mathcal{N}(\star{x}_n, \star{\Sigma}_n)$ } 
% \subsection{Merging independent measurements}
\label{sec:combining}
\label{sec:inter}
Given the probabilistic interpretation in \Section{independent}, a \emph{joint} optimization over parameters can be obtained by cumulating the posterior distributions over the set of previous frames. This leads to the to the following pair of inductive update equations:
% 
\begin{align}
\mathcal{N}(\hat{x}_1, \hat{\Sigma}_1) &= \mathcal{N}(\star{x}_1, \star{\Sigma}_1) \\
\mathcal{N}(\hat{x}_n, \hat{\Sigma}_n) &= \mathcal{N}(\hat{x}_{n-1}, \hat{\Sigma}_{n-1}) \mathcal{N}(\star{x}_n, \star{\Sigma}_n)
\end{align}
% 
by applying the product of Gaussians rule ~\cite{petersen2008matrix}:
% 
\begin{align}
\begin{split}
\hat{x}_{n} &= \todo{\Sigma_{n}} (\hat{\Sigma}_{n-1} + \Sigma_{n})^{-1} \hat{x}_{n-1} + 
\hat{\Sigma}_{n-1} (\hat{\Sigma}_{n-1} + \Sigma_n)^{-1} x_n^*
\\
\hat{\Sigma}_n &= \hat{\Sigma}_{n-1} (\hat{\Sigma}_{n-1} + \todo{\Sigma_n})^{-1} \todo{\Sigma_n}
\label{eq:combining}
\end{split}
\end{align}
% 
In \Appendix{kalman}, we shown how \Equation{combining} is equivalent to the Kalman filter update equations in \Table{interframe}, with measurement $x_n^*$, and measurement noise covariance \todo{$\Sigma_n$}. This observation is fundamental as, under the same assumptions made in \Appendix{kalman}, \cite{maybeck1979stochastic} noted how a Kalman filter is provably \emph{optimal} for the problem at hand.
% ``the Kalman filter can be shown to be the best filter of any conceivable form. It incorporates all information that can be provided to it.''}
This optimization will be used as a \emph{baseline}, as it is arguably the simplest way of achieving an online parameter regression by treating the results of the independent solve $\mathcal{N}(\star{x}_n, \star{\Sigma}_n)$ as the measurements in a KF.

\input{tab/iekf-like}
\subsection{Joint inter/intra frame regression}
% Note that the best-fitting parameters $x_n^*$ for the given the single input frame $d_n$ are computed using local LM optimization. Thus, it is crucial that optimization starts in a sensible region of parameters space.
The optimization in \Tab{interframe} does not provide any information about the current estimate of the parameters $\hat{x}_n$ to the independent solve described in \Section{independent}. This could be problematic, as in this case \Eq{energies} does not leverage any temporal information, and relies on a sufficiently good initialization to compute $\mathcal{N}(\star{x}_n, \star{\Sigma}_n)$. One potential way to increase the robustness of the algorithm would be to include this information in \Eq{independent}:
% 
\begin{align}
x^* &= \argmin_{x} \log  P(d|x) P(x_n |\hat{x}_{n - 1}) 
\label{eq:independenttemp}
\end{align}
% 
However, rather than following this path (\TODO{need a better argument... quantitative comparison?}), we coalesce the inter and intra frame optimization, and propose a \emph{joint} intra/inter online regression scheme in \Table{iekf-like}. Interestingly, we demonstrate that optimizing the objective in~\Table{iekf-like} with a Levemberg-Marquardt optimization is equivalent to an Iterated Extended Kalman Filter (IEKF), with \todo{measurement update... what?}.
Notice how this optimization, representing our online tracking\&modeling system, jointly considers the measurement $d_n$ as well as past estimates $\mathcal{N}(\hat{x}_n, \hat{\Sigma}_n)$.
\newpage

\begin{figure}[b]
\centering
\begin{overpic} 
[width=\linewidth]
% [width=\linewidth,grid,tics=10]
{\currfiledir/item.png}
\myfigurename{}
\end{overpic}
\caption{
% 
We estimate the parameter $\mu$ from two measurements $\mu_1$ and $\mu_2$; these measurements are associated with uncertainty ellipsoids represented as their covariance matrices $\cov_1$ and $\cov_2$. The information in these two estimates can be compound, resulting in the joint estimate $\tilde{\mu}$. \AT{why the equations above look different from \Eq{combining}?} \AT{we might want to move this figure to the appendix too. It's quite basic, as it's just a visualization of the product rule of two gaussians, no? Or is there something else you'd like to say anastasia?}
% 
}
\label{fig:merging}
\end{figure}

\subsection{Shape prior (TODO)}
\label{sec:shapeprior}
\TODO{Anastasia?}
\AN{The shape prior should only be used in the first frame, because afterwards it will be incorporated inside of Kalman prior}. \AT{As this is quite advanced, this can be discussed at the very end of the section}
