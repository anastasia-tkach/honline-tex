\section{Technical}
Given the $n$-th input depth frame $\mathcal{D}_n$ we first segment the hand point cloud~\cite{htrack}, and represent it in vectorized form as $d_n$. With $x_n = [\theta_n; \beta_n]$ we represent vector of coalesced pose and shape parameters at frame $n$. To achieve online joint calibration and tracking, our algorithm presents two fundamental components which we refer to as \emph{intra-} and \emph{inter-frame} regression.

\paragraph{Intra-frame}
Given a proper initialization $x_n^0$ and the input data $d_n$, intra-frame regression solves for a locally optimal $x_n^*$, as well as estimates the \emph{uncertainty} in the given solution. That is, this scheme provides a distribution for $x_n$ that is solely based on the knowledge in the depth frame $\mathcal{D}_n$. In~\Section{independent}, following the ideas in ~\todo{\cite{?}}, we derive how this distribution can be approximated by a gaussian $\mathcal{N}(\hat{x}_n, \hat{\Sigma}_n)$, and how the parameters of this distribution can be \emph{directly} obtained from the Jacobians of the tracking optimization. \todo{The intuition is in... \Figure{uncertainty}}.

\paragraph{Inter-frame}
\TODO{describe how to combine}

\todo{Given the hand point cloud at the first frame $d_1$, we find the best-fitting parameters, and then derive a quadratic approximation of their posterior distribution; see Section \ref{sec:independent}.} 

\AN{The shape prior should only be used in the first frame, because afterwards it will be incorporated inside of Kalman prior}.
\AT{As this is quite advanced, this can be discussed at the very end of the section}


\subsection{Intra-frame regression -- $\mathcal{N}(\hat{x}_n, \hat{\Sigma}_n)$}
\label{sec:independent}
% 
Given $d_n$ the point cloud at frame $n$, and an initialization of hand pose from the previous frame $x_n^0 = x_{n - 1}^*$ we find an independent estimate of $x_n^*$ by local iterative optimization (a-la Levenberg) of the multi-objective energy function:
% 
\begin{equation}
x_n^* = \argmin_{x_n} \sum_{\tau \in \mathcal{T}} E_{\tau}(d_n, x_n) 
\label{eq:energies}
\end{equation}
% 
Where the terms $\mathcal{T}$ ensure that:
%
% \vspace{-.5\parskip}
\begin{equation*}
\setlength{\jot}{0pt}
\begin{aligned}
\text{\textbf{d2m}} & \quad \text{data points are explained by the model} \\ 
\text{\textbf{m2d}} & \quad \text{the model lies in the sensor visual-hull} \\
\text{\textbf{shape-prior}} & \quad \text{the model shape should be plausible} \\
\text{\textbf{pose-prior}} & \quad \text{the model pose should be plausible} \\
\text{\textbf{collision}} & \quad \text{fingers do not interpenetrate} \\
\text{\textbf{limits}} & \quad \text{joint limits are satisfied}
\end{aligned}
\end{equation*}
The energy terms in the objective function are detailed in~\cite{tkach2016sphere} and~\cite{htrack}, with the exception of \emph{shape-prior} that is discussed in \Section{shapeprior}.

% \subsection{Posterior distribution of parameters after taking a measurement into account}
\begin{figure}[t]
\centering
\begin{overpic} 
[width=\linewidth]
% [width=\linewidth,grid,tics=10]
{\currfiledir/item.png}
\myfigurename{}
\put(15,-3){straight finger}
\put(65,-3){bent finger}
\end{overpic}
\caption{
We abstract the hand shape/pose estimation problem into the one of a simpler 2D stick-figure.  When the finger is straight (left), it is difficult to estimate finger length as the optimization problem is ill-posed, while with a bent finger (right) the problem is better conditioned.
% 
We analyze the energy landscape of the registration problem $E(\shape_1)$, and observe how estimation uncertainty relates to the width of the local minima valley. This certainty can be estimated through a quadratic approximation to $E(\shape_1)$, derived from the Jacobians of the registration energies.
%  
}
\label{fig:uncertainty}
\end{figure}

\paragraph{Laplace approximation}
\label{sec:posterior}
% 
\TODO{mention why we do this, and refer to \Figure{uncertainty} to explain, and mention \todo{\cite{laplace}} was the first to do this}
We rewrite each energy term as: 
% in \Equation{energies} as:
\begin{align}
E_{\tau}(d_n, x_n) = \|I_{\tau} d_n - F_{\tau} (x_n)\|_2^2,
\end{align} 
where most $I_{\tau}$ are zero matrices, $I_{d2m}$ and $I_{m2d}$ are identity matrices, and $I_{m2d}(i,i)=0$ if the rasterized 2D template pixel $\mathbf{p}_i$ lies within the sensor silhouette image \AT{correct?}; see \cite{htrack}.
% energy terms  can be rewritten in a form $$ with ,  a permutation(/reduction?) matrix that removes from $d_n$ the \todo{entries that are not the closest for any model point} \AT{meaning?},
For brevity of notation, we also denote:
% 
\begin{align*}
F(x_n) &= \left[F_{d2m}^T(x_n), ..., F_{pose}^T(x_n)\right]^T \\
d_n &= \left[(I_{d2m} d_n)^T, ..., (I_{pose} d_n)^T\right]^T
\end{align*}
% 
leading to a compact posterior distribution representation:
% 
\begin{align}
P(d_n|x_n) &= \exp \left( - \tfrac{1}{2}(d_n - F(x_n))^T (d_n - F(x_n)) \right)
\label{eq:posterior}
\end{align}
% 
\Equation{energies} is then rewritten in probabilistic form, where we temporarily omit the frame index $n$ for conveniency:
%
\begin{align}
x^* &= \argmin_{x} \underbrace{\log  P(d|x)}_{L(x)}
\label{eq:independent}
\end{align}
%
We then perform a second-order Taylor expansion of the log-likelihood of the data $L(x)$ around the \emph{optimal} solution $x^*$:
%
\begin{align*}
L(x) \approx \tilde{L}(x) = L(x^*)   
+ \left. \tfrac{\partial L(x) }{\partial x}\right|_{x^*}  \Delta x 
+ \left. \tfrac{1}{2} \Delta x^T\tfrac{\partial \partial L(x)}{\partial x \partial x} \right|_{x^*} \Delta x + \text{h.o.t.}
\end{align*}
%
where $\Delta x=x - x^*$, and Jacobian and Hessian are respectively zero and positive definite at our critical point $x^*$:
%
\begin{align}
\left.\tfrac{\partial L(x)}{\partial x} \right|_{x^*} &= - 2 F(x^*)^T 
\left.\tfrac{\partial F(x)}{\partial x}\right|_{x^*} = 0 
\label{eq:taylor-jacobian}
\\
\Sigma = \left. \tfrac{\partial \partial L(x)}{\partial x \partial x}  \right|_{x^*}  
% = 2 \left.\tfrac{\partial F(x)}{\partial x}\right|_{x^*}^T \left.\tfrac{\partial F(x)}{\partial x}\right|_{x^*} + \\ 2 F(x^*)^T  \left. \tfrac{\partial \partial F (x^*) }{\partial \partial x} \right|_{x^*}
& \approx 2
\left.\tfrac{\partial F}{\partial x}\right|_{x^*}^T  
\left.\tfrac{\partial F}{\partial x}\right|_{x^*}
\succ 0
\label{eq:taylor-hessian}
\end{align}
% 
Remembering that $\tilde P(d|x) = \exp (\tilde{L}(x))$, we can then derive the \emph{approximated} posterior distribution:
%
\begin{align}
\tilde{P}(d|x) = \exp\left(- \tfrac{1}{2}(x - x^*)^T \Sigma^{-1}  (x - x^*) \right) = \mathcal{N}\left(\hat{x}^*, \Sigma \right)
\end{align}
%
That is, after processing the information in a frame $d_n$, the desired quadratic approximation of posterior distribution of model parameters is a \emph{normal} distribution $\mathcal{N}\left(\hat{x}_n, \hat{\Sigma}_n\right)$.
Its mean $\hat{x}_n = x^*$ is the (final) solution of the (iterative) optimization problem in \Equation{energies}; its variance $\hat{\Sigma}_n = \Sigma$ is computed according to \Equation{taylor-jacobian}, that is, from the Jacobian computed in the last iteration of~\Equation{energies}. \AT{we might have to also write the iterative version of \Equation{energies} as I did in Edoardo's paper to make this clear}.

\newpage
% \subsection{Merging independent measurements}
\subsection{Inter-frame regression} 
\label{sec:combining}
\begin{figure}[b]
\centering
\begin{overpic} 
[width=\linewidth]
% [width=\linewidth,grid,tics=10]
{\currfiledir/item.png}
\myfigurename{}
\end{overpic}
\caption{
% 
We estimate the parameter $\mu$ from two measurements $\mu_1$ and $\mu_2$; these measurements are associated with uncertainty ellipsoids represented as their covariance matrices $\cov_1$ and $\cov_2$. The information in these two estimates can be compound, resulting in the joint estimate $\tilde{\mu}$. \AT{why the equations above look different from \Eq{combining}?} \AT{we might want to move this figure to the appendix too. It's quite basic, as it's just a visualization of the product rule of two gaussians, no? Or is there something else you'd like to say anastasia?}
% 
}
\label{fig:merging}
\end{figure}


Given the second frame $d_2$, we compute the corresponding value of the parameters $x_2^*$ and their variance $\Sigma_2$ the same way as in \ref{sec:posterior}. The posterior distribution of the parameters now has the form:
% 
\begin{equation}
\mathcal{N}(\hat{x}_2, \hat{\Sigma}_2) = \mathcal{N}(\hat{x}_1, \hat{\Sigma}_1) \mathcal{N}(x_2^*, \Sigma_2) \quad\quad
\end{equation}
% 
applying the product of Gaussians rule ~\cite{petersen2008matrix}, we obtain:
% 
\begin{align}
\begin{split}
\hat{x}_2 &= \Sigma_2 (\hat{\Sigma}_1 + \Sigma_2)^{-1} \hat{x}_1 + 
\hat{\Sigma}_1 (\hat{\Sigma}_1 + \Sigma_2)^{-1} x_2^*
\\
\hat{\Sigma}_2 &= \hat{\Sigma}_1 (\hat{\Sigma}_1 + \Sigma_2)^{-1} \Sigma_2
\label{eq:combining}
\end{split}
\end{align}

In Appendix \ref{app:kalman}, we shown how \Equation{combining} is equivalent to Kalman filter update with measurement $x_n^*$ and measurement noise covariance $\Sigma_n$ summarized in \Figure{interframe}.

\AN{This is important, because, according to P. Maybeck \cite{maybeck1979stochastic}, under the assumptions presented in Appendix \ref{app:kalman} ``the Kalman filter can be shown to be the best filter of any conceivable form. It incorporates all information that can be provided to it.''}

%%%%%%%%%%%%%%%%%%%%%%%%%%%%%%%%%%%%%%%%%%%%
\begin{table}[ht] 
\centering
\caption{KF: Combining independent measurements\label{tab:kf-like}} 
\begin{tabular}{|c|}
\hline
$x_n^* = \argmax_{x_n} \log P(d_n|x_n)$ \\
$\hat{x}_n = \argmax_{x_n} \underbrace{\log \left( P(x_n|x_n^*) P(x_n |\hat{x}_{n - 1}) \right)}_{L(x_n)}$\\
with \\
$P(x_n |x_n^*) = \exp \left( - (x_n - x_n^* )^T \Sigma_n^{-1}(x_n - x_n^* )\right)$ \\
$P(x_n |\hat{x}_{n - 1}) = \exp \left( - (x_n - \hat{x}_{n - 1} )^T \hat{\Sigma}_{n - 1}^{-1} (x_n - \hat{x}_{n - 1} )\right)$ \\	

$\hat{\Sigma}_n^{-1} = \left. \tfrac{\partial \partial L(x_n)}{\partial x_n \partial x_n}\right|_{\hat{x}_n} \approx $ \\
\\
$\left[
	\begin{array}{cc}
		\left(\Sigma_n^{-1}\right)^{1/2} \\
		\left(\hat{\Sigma}_{n - 1}^{-1}\right)^{1/2} \\
	\end{array}
\right]^T 
\left[
	\begin{array}{c}
		\left(\Sigma_n^{-1}\right)^{1/2} \\
		\left(\hat{\Sigma}_{n - 1}^{-1}\right)^{1/2} \\
	\end{array}
\right] = \hat{\Sigma}_{n-1}^{-1}  + \Sigma_{n}^{-1}$ \\

\hline
\end{tabular}
\end{table}
%%%%%%%%%%%%%%%%%%%%%%%%%%%%%%%%%%%%%%%%%%%%

\subsection{Regularizing optimization}

Note that the best-fitting parameters $x_n^*$ for the given the single input frame $d_n$ are computed using local LM optimization. Thus, it is crucial that optimization starts in a sensible region of parameters space. The optimization presented in Table \ref{tab:kf-like} does not provided any information about the current estimate of the parameters $\hat{x}_n$ to the independent solve from Section \ref{sec:independent}. One way to remedy this is include the term $P(x_n |\hat{x}_{n - 1})$ directly in the independent solve.

%%%%%%%%%%%%%%%%%%%%%%%%%%%%%%%%%%%%%%%%%%%%
\begin{table}[ht] 
\centering
\caption{Objective function of IEKF \label{tab:iekf-like}} 
\resizebox{0.48\textwidth}{!}{
\begin{tabular}{|c|}

\hline

$\hat{x}_n = \operatorname{argmax}_{x_n} 
\underbrace{\log \left( P(d_n|x_n) P(x_n |\hat{x}_{n - 1}) \right)}_{L(x_n)}$\\ 
with \\
$P(d_n|x_n) = \exp \left( \tfrac{1}{2} - (d_n - F(x_n))^T (d_n - F(x_n)) \right)$ \\
$P(x_n |\hat{x}_{n - 1}) = \exp \left( \tfrac{1}{2} -(x_n - \hat{x}_{n - 1})^T \hat{\Sigma}_{n - 1}^{-1} (x_n - \hat{x}_{n - 1})\right)$ \\
\\
$\hat{\Sigma}_n^{-1} =  \left. \tfrac{\partial \partial L(x_n)}{\partial x_n \partial x_n}\right|_{\hat{x}_n} \approx $ \\
\\
$\left[
	\begin{array}{cc}
		- \left. \tfrac{\partial F(x_n)} {\partial x_n} \right|_{\hat{x}_n} \\
		\left(\hat{\Sigma}_{n - 1}^{-1}\right)^{1/2} \\
	\end{array}
\right]^T 
\left[
	\begin{array}{c}
		-\left. \tfrac{\partial F(x_n)} {\partial x_n} \right|_{\hat{x}_n} \\
		\left(\hat{\Sigma}_{n - 1}^{-1}\right)^{1/2} \\
	\end{array}
\right] = $\\
\\
$ = \hat{\Sigma}_{n-1}^{-1} + \left. \tfrac{\partial F(x_n)} {\partial x_n} \right|_{\hat{x}_n}^T
\left. \tfrac{\partial F(x_n)} {\partial x_n} \right|_{\hat{x}_n} = \hat{\Sigma}_{n-1}^{-1}  + \Sigma_{n}^{-1}$	\\
\hline
\end{tabular}
}
\end{table}

In Appendix \ref{app:ieif-lm} we demonstrate that optimizing the objective function presented in Table \ref{tab:iekf-like} using Levenberg-Marquardt algorithm is equivalent to measurement update of Iterated Extended Kalman Filter.

\subsection{Comparing the two optimizations}
The optimization presented in Table \ref{tab:kf-like} will be used as a baseline. It treats the results of the independent solve $x_n^*$ as a measurement. This approach, arguably can be considered to be the simplest way of maintaining an online parameters estimate. 
The optimization shown in Table \ref{tab:iekf-like} represents the entire hand tracking system as an Iterated Extended Kalman Filter. Given the next measurement $d_n$, the estimated value of the parameters $\hat{x}_n$,  is directly output by the hand tracking system.

\subsection{Shape prior}
\label{sec:shapeprior}
\TODO{Anastasia?}
