\definecolor{truth}{RGB}{180,180,180}
\definecolor{intra}{RGB}{188,58,117}
\definecolor{inter}{RGB}{61,131,119}
\definecolor{varintra}{RGB}{227,176,171}
\definecolor{varinter}{RGB}{160,215,190}

% 
\begin{figure*}[t]
\centering
\begin{overpic} 
[width=\linewidth]
% [width=\linewidth,grid,tics=10]
{\currfiledir/item.pdf}
\myfigurename{}
\put(0.5,17){\scriptsize \vertical{$E_\beta$}}
\put(98,17){\scriptsize \vertical{$\Sigma^{-1}$}}
\put(7,-1){\scriptsize thumb middle length -- \todo{$\beta_{[?]}$}}
\put(32,-1){\scriptsize ring attachment y -- \todo{$\beta_{[?]}$}}
\put(55,-1){\scriptsize middle middle length -- \todo{$\beta_{[?]}$}}
\put(80,-1){\scriptsize middle attachment z -- \todo{$\beta_{[?]}$}}
\end{overpic}
\caption{
% 
We illustrate the formulation in \Section{technical} on the calibration of four different degrees-of-freedom on the \texttt{Handy/Teaser} dataset. 
(top)
% 
Frames are indexed by $n$ and we display: 
{\color{truth} ground-truth value $\bar{\beta}$},
{\color{intra} intra-frame estimate $\star{\beta}_n$} and 
{\color{inter} inter-frame estimate $\hat{\beta}_n$}; the scale for these quantities is relative to $\bar{\beta}$, and shown on left-hand side of each axis.
% 
In the same plot, we also visualize the \emph{inverse} of the diagonal entry of the covariance matrices (i.e. certainty) for {\color{varintra} intra-frame $\star\Sigma_n$} and {\color{varinter} inter-frame $\hat\Sigma_n$} estimates; the scale for these quantities is on the right-hand side of each plot.
% 
(bottom) We also display the pose corresponding to a selection of frames in the sequence~(small grey arrows); please note the correlation between pose and estimate uncertainty.
% 
\AT{might not be a bad idea to only have 3x figures here? Otherwise the plots are very very small. Plus I think we need to fill the extra page anyway}
% 
}
\label{fig:interreal}
\end{figure*}

\section{Evaluation}
To evaluate the technical validity of our approach we:
(\Sec{evalsynth}) analyze its robustness through randomly perturbing the algorithm initialization; 
(\Sec{analysis}) corroborate the formulation of our optimization on a synthetic 3D dataset;
(\Sec{evaldataset}) verify its effectiveness on by applying our algorithm to a new calibration dataset acquired through a commodity depth camera; 
(\Sec{evalstar}) attest how our method achieves state-of-the-art performance on publicly available datasets.

\newpage
\subsection{Synthetic dataset: robustness -- \Fig{synthetic}}
\label{sec:evalsynth}
We empirically evaluate the \emph{robustness} of the algorithm by analyzing its convergence properties. 
For synthetic data the ground truth shape parameters $\bar\beta$ are readily available, and the sphere-mesh model $\mathcal{M}(\theta,\beta)$ is animated in time with the $\bar\theta_n$ parameters of the complex motions in the \texttt{Handy/Teaser} dataset~\cite{tkach2016sphere}.
We employ two metrics, one measuring average ground truth residuals, the other measuring the average length of data-to-model ICP correspondences (i.e. part of the tracking energy):
% 
\begin{equation*}
E_{\beta} = \tfrac{1}{|M|} \sum_{m \in M} \left| \beta_m - \bar\beta_m \right|, 
\:
E_\text{d2m}^n = \tfrac{1}{|d_n|} \sum_{\mathbf{p}_j \in d_n} \| \mathbf{p}_j - \Pi_{\mathcal{M}(\theta,\beta)}(\mathbf{p}_j) \|_2
\label{eq:metrics}
\end{equation*}
% 
Note how we selected only a subset \todo{$M=\{1,2,3,4\}$} of shape parameters; see \Figure{handmodel}. This is necessary as sphere-centres on the palm can move without affecting the tracking energy -- a null-space of our optimization.
% 
The tracking algorithm is initialized in the first frame by $\bar\theta_0$, while initializations for the user-personalized models are drawn from the uniform distribution {\small $\mathcal{U}(\bar\beta, \sigma)$}.
We limit ourselves to ten samples per value of $\sigma$, as each sample requires the re-execution of our algorithm on an entire sequence. Nonetheless, we empirically noticed that increasing the number of samples did not alter our observations.
% \AT{I still REALLY don't like this... It is not statistically significant. We'd run at least 50-100 samples...}
In \Figure{synthetic}, we visualize our two metrics as we vary the intensity of the perturbation. Our perturbation analysis is performed over \todo{2h30s} of synthetic footage, thus in our \VideoSynth{}, we only report a \emph{random selection} of tracked sequences.
% 

\subsection{Synthetic dataset: formulation analysis -- \Fig{interreal}}
\label{sec:analysis}
In \Figure{interreal}, we report an experiment analogous to that of \Figure{inter} but on a full 3D sequence. For example, consider the {\small \todo{$\beta_{[?]}$}} plot, where we report the runtime estimates for the length of the middle-finger's middle-phalanx (to avoid confusion with frame number $n$, the \todo{$[?]$} subscript is implied).
As expected, the phalanx length can be estimated only in those frames where the finger is bent, as the callout renderings show.
%
Notice how the intra-frame estimate {\small $\star\beta_n$} can oscillate away from the ground truth, and such incorrect estimates should be considered. To achieve this, our algorithm estimates the solution uncertainty {\small $\Sigma^*_n$}, and accordingly updates the online cumulative estimates {\small $\hat\beta_n$} and {\small $\hat\Sigma_n$}.

\begin{figure}[b]
\centering
\begin{overpic} 
[width=\linewidth]
% [width=\linewidth,grid,tics=10]
{\currfiledir/item.pdf}
\myfigurename{}
\end{overpic}
\caption{
% 
Evaluation on the \emph{handy} dataset from \protect\cite{tkach2016sphere}.
% 
}
\label{fig:evalhandy}
\end{figure}

\subsection{Calibration dataset: \texttt{GuessWho?} -- \Fig{realtrack}}
\label{sec:evaldataset}
We stress-tested our system by qualitative evaluating our calibration technique with data acquired from twelve different users performing in front of an Intel RealSense SR300 camera (time-of-flight depth RGBD sensor). 
Snapshot of the twelve calibration sequences are reported in \Figure{realtrack}, but the calibration procedure can be better appreciated by watching our \VideoQualitative{}.
While ground truth information is not available, these datasets can be evaluated through the use of empirical metrics; e.g. $E_\text{d2m}$ and $E_\text{m2d}$ from \cite{tkach2016sphere}, or the golden-energy from~\cite{taylor2016joint}. Hence, to foster research in this domain, we release these calibration sequences to the community as the ``\texttt{GuessWho?}'' dataset. \AT{get the pun? :P}
 
\begin{figure}[b]
\centering
\begin{overpic} 
[width=\linewidth,height=3in]
% [width=\linewidth,grid,tics=10]
{\currfiledir/item.pdf}
\myfigurename{}
\end{overpic}
% \vspace{-.3in}
\caption{
% 
Evaluation on the NYU dataset from \protect\cite{tompson2014real}.
% 
}
\label{fig:evalnyu}
\end{figure}

\subsection{State-of-the-art on $\{$\texttt{Handy, NYU}$\}$ --
\Fig{evalhandy} and \Fig{evalnyu}}
\label{sec:evalstar}
We quantitatively evaluate our algorithm in terms of tracking performance on the NYU~\cite{tompson2014real} and Handy~\cite{tkach2016sphere} datasets. \AT{WORK IN PROGRESS...}
% 
\begin{DRAFT}
Lorem ipsum dolor sit amet, consectetur adipisicing elit, sed do eiusmod tempor incididunt ut labore et dolore magna aliqua. Ut enim ad minim veniam, quis nostrud exercitation ullamco laboris nisi ut aliquip ex ea commodo consequat. Duis aute irure dolor in reprehenderit in voluptate velit esse cillum dolore eu fugiat nulla pariatur. Excepteur sint occaecat cupidatat non proident, sunt in culpa qui officia deserunt mollit anim id est laborum.
Lorem ipsum dolor sit amet, consectetur adipisicing elit.
\end{DRAFT}
