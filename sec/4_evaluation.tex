\section{Evaluation}
To evaluate the technical validity of our approach we verify its effectiveness on by applying it to a new calibration dataset acquired through commodity depth cameras (\Sec{evaldataset}); corroborate the formulation of our optimization on a synthetic 3D dataset (\Sec{analysis}); analyze its robustness through randomly perturbing the algorithm initialization (\Sec{evalsynth}); and attest how our method achieves state-of-the-art performance on publicly available datasets (\Sec{evalstar}).

\subsection{Synthetic dataset: formulation analysis -- \Fig{interreal}}
\label{sec:analysis}
For synthetic data the ground truth shape parameters $\bar\beta$ are readily available, and the sphere-mesh model $\mathcal{M}(\theta,\beta)$ is animated in time with the $\bar\theta_n$ parameters of the complex motions in the \texttt{Handy/Teaser} dataset~\cite{tkach2016sphere} \AN{It is not on Handy/Teaser, that sequence starts from fast motion where sensor data sets destroyed, which is bad for calibration.}.
The model $\mathcal{M}(\theta,\beta)$ is rendered with the camera intrinsics producing a depth map $\mathcal{D}(\theta,\beta)$. We employ two metrics, one measuring average ground-truth residuals, the other measuring the average length of data-to-model ICP correspondences (i.e. part of the tracking energy):
% 
\begin{align}
E_{\beta} &= \tfrac{1}{|M|} \sum_{m \in M} \left| \beta_{[m]} - \bar\beta_{[m]} \right|
\label{eq:metricgt}
\\
E_\text{d2m}^n &= \tfrac{1}{|d_n|} \sum_{\mathbf{p}_j \in d_n} \| \mathbf{p}_j - \Pi_{\mathcal{D}(\theta,\beta)}(\mathbf{p}_j) \|_2
\label{eq:metricd2m}
\end{align}
% 
Where $\Pi$ is an operator computing the closest-point projection of points in the sensor's point cloud, onto the point-cloud associated with the synthetic depth-map $\mathcal{D}(\theta,\beta)$; see~\cite{tkach2016sphere}. For ground-truth comparisons, we selected only a subset \TODO{$M=\{1,2,3,4\}$} of shape parameters; see \Figure{handmodel}. This is necessary as sphere-centres on the palm can move without affecting the tracking energy -- a null-space of our optimization. The tracking algorithm is initialized in the first frame by $\bar\theta_0$.
%
$\quad$
% 
In~\Figure{interreal},~we report an experiment analogous to that of \Figure{inter} but on a full 3D sequence. For example, consider the {\small \TODO{$\beta_{[?]}$}} plot, where we report the runtime estimates for the length of the middle-finger's middle-phalanx (to avoid confusion with frame number $n$, the subscript~\TODO{{\small$[?]$}}~is implied). Congruously to the discussion in \Section{technical}, the phalanx length \AN{estimates from the frames where the finger is bent are given large weight,}
%can be estimated only in those frames where the finger is bent; \AN{this is not the case}
see callout renderings.
Notice how the intra-frame estimate {\small $\star\beta_n$} often oscillates away from the ground truth: 
such incorrect estimates \AN{have small weight.}
%should be discarded while accumulating information.%
 To achieve this, our algorithm estimates the solution uncertainty {\small $\Sigma^*_n$}, and accordingly updates the online cumulative estimates~{\small $\hat\beta_n$}~and~{\small $\hat\Sigma_n$}.

\subsection{Synthetic dataset: robustness -- \Fig{synthetic}}
\label{sec:evalsynth}
We evaluate the \emph{robustness} of the algorithm by analyzing its convergence properties. Random initializations for the user-personalized models are drawn from the uniform distribution {\small $\mathcal{U}(\bar\beta, \sigma)$}.
We limit ourselves to ten samples per value of $\sigma$, as each sample requires the re-execution of our algorithm on an entire sequence. Nonetheless, we noticed that increasing the number of samples did not alter our observations.
% \AT{I still REALLY don't like this... It is not statistically significant. We'd run at least 50-100 samples...}
In \Figure{synthetic}, we visualize our two metrics as we vary the intensity of the perturbation.
 Our perturbation analysis is performed over \TODO{2h30s} of synthetic footage, thus in our \VideoSynth{}, we only report a \emph{random selection} of tracked synthetic sequences.\AN{The video of the synthetic data that you saw if not a random selection of the sequences, it is a single sequence where the model is randomly perturbed every several frames. This is not what happens during evaluation. But the actual evaluation would be boring to watch.}


\subsection{Calibration dataset: \texttt{GuessWho?} -- \Fig{realtrack}} 
\label{sec:evaldataset}
We stress-tested our system by qualitatively evaluating our calibration technique with data acquired from \emph{twelve} different users performing in front of an Intel RealSense SR300 camera (a consumer-level time-of-flight depth sensor). Snapshots of the twelve calibration sequences are reported in \Figure{realtrack}.
% , but the calibration procedure can be better appreciated by watching our \VideoQualitative{}.
While ground truth information is not available, these datasets can be evaluated through the use of empirical metrics; e.g. $E_\text{d2m}$ and $E_\text{m2d}$ from \cite{tkach2016sphere}, or the golden-energy from~\cite{taylor2016joint}.
Hence, to foster research in this domain, we release these calibration sequences to the community as the ``\texttt{GuessWho?}'' dataset.
% -- future works will be able to generate identical synthetic clones of human hands (metrics~$\approx 0$), and thus enable biometric authentication applications.

\begin{figure}[b]
\centering
\begin{overpic} 
[width=\linewidth]
% [width=\linewidth,grid,tics=10]
{\currfiledir/item.pdf}
\myfigurename{}
\put(0,51){\scriptsize \vertical{$E_\beta$}}
\put(83,1){\scriptsize $\sigma$}
\end{overpic}
\caption{
% 
Evaluation on synthetic dataset:
% 
(left) random perturbation with standard deviation $\sigma$ vs. ground truth calibration residuals;
% 
(right) a few (up-to-scale) exemplars drawn from the $\sigma=0.4$ perturbation.
% 
The solid line represent the mean, and the dotted line represents the empirical variance.
% 
}
\label{fig:synthetic}
\end{figure}

\subsection{Marker-based evaluation on $\texttt{NYU}$ dataset --
\Fig{evalnyu}}
\label{sec:evalstar}
Although several marker-based datasets are available, such as \cite{qian2014realtime}, \cite{sharp2015accurate} and \cite{yuan2017bighand}, state-of-the-art \emph{generative} methods have focused on the \texttt{NYU} \cite{tompson2014real} and \texttt{Handy} \cite{tkach2016sphere} datasets for quantitative evaluation. 
\TODO{anastasia, I need to know which joints were used? no palm, only fingers, etc...} \AN{6x4 markers from each finger and 4 markers from the thumb, palm and wrist were not used}
On the \texttt{NYU} dataset, to properly compare to~\cite{taylor2016joint}, we evaluate the metrics on the first 2440 frames (user \#1). 
This dataset allow us to compare our method (and its variants) to a number of other algorithms including: the PSO tracker by~\cite{sharp2015accurate}, the calibration methods by~\cite{khamis2015learning} and \cite{tan2016fits}, the subdivision tracker of~\cite{taylor2016joint}, the cylindroid tracker by~\cite{htrack}, the sphere-mesh tracker by~\cite{tkach2016sphere}, as well as the Gaussian tracker in~\cite{sridhar2015fast}, and finally discriminative methods such as~\cite{tompson2014real}, \cite{tang2015opening} and \cite{oberweger2015hands}. 
% 
% \TODO{describe scaled template}
% 
Our online algorithm achieves very competitive tracking performance while being the \emph{first} capable to calibrate the user-personalized tracking model \emph{online}, rather than in an offline calibration session like~\cite{taylor2016joint}. 
% 
Notice how the best performance is achieved by either (1) the intra-frame optimization, where per-frame \emph{overfitting} takes place, or (2) by offline calibration techniques such as \OfflineSoft{} or \cite{taylor2016joint} having a clear advantage (i.e. infinite memory) over streaming algorithms.  \AN {I am confused about what are you trying to say with \lq\lq{}infinite memory\rq\rq{}}

\begin{figure}[b]
\centering
\begin{overpic} 
[width=\linewidth,height=3in]
% [width=\linewidth,grid,tics=10]
{\currfiledir/item.pdf}
\myfigurename{}
\end{overpic}
% \vspace{-.3in}
\caption{
% 
Evaluation on the NYU dataset from \protect\cite{tompson2014real}.
% 
}
\label{fig:evalnyu}
\end{figure}

\begin{figure}[b]
\centering
\begin{overpic} 
[width=\linewidth]
% [width=\linewidth,grid,tics=10]
{\currfiledir/item.pdf}
\myfigurename{}
\end{overpic}
\caption{
% 
Evaluation on the \emph{handy} dataset from \protect\cite{tkach2016sphere}.
% 
}
\label{fig:evalhandy}
\end{figure}

\subsection{Dense evaluation on the $\texttt{Handy}$ dataset -- \Fig{evalhandy}}
Another way to evaluate the quality of tracking/calibration is to compare the depth map $\mathcal{\bar{D}}_n$ (i.e. sensor point cloud) to the rendered depth map $\mathcal{D}(\theta,\beta)$ (i.e. model point cloud). As far as rendered depth-maps for the posed and personalized tracking model are available, the metric in \Equation{metricd2m} can be employed to evaluate \emph{dense}, as opposed to marker-based, tracking/calibration performance comparisons: if $E_\text{d2m}\approx0$ in every frame (up to sensor noise), then the personalized model is a perfect \emph{dynamic} digital replica of the user's hand.
The \texttt{Handy} dataset from~\cite{tkach2016sphere} enables these type of comparisons and includes rendered depth maps for \cite{htrack}, \cite{sharp2015accurate}, as well as the state-of-the-art method of~\cite{taylor2016joint}. Note how this dataset considers a range of motion substantially more complex than the one in the \texttt{NYU} dataset.
Like in earlier comparisons, the intra-frame technique performs best as it overfits to the data (as it generates a collection of $\beta_n$ instead of a single $\beta$). Our techniques calibrate a model with performance comparable to that of~\cite{tkach2016sphere}, where a high-quality multi-view stereo point cloud with manual annotations was used for calibration. \TODO{anastasia, can generate two more curves for batch comparisons? \textbf{absolutely} needed}
