\section{Evaluation}
To evaluate the technical validity of our approach we:
(\Sec{evalsynth}) analyze its robustness through randomly perturbing the algorithm initialization; 
(\Sec{analysis}) corroborate the formulation of our optimization on a synthetic 3D dataset;
(\Sec{evaldataset}) verify its effectiveness on by applying our algorithm to a new calibration dataset acquired through a commodity depth camera; 
(\Sec{evalstar}) attest how our method achieves state-of-the-art performance on publicly available datasets.

\subsection{Synthetic dataset: robustness -- \Fig{synthetic}}
\label{sec:evalsynth}
We empirically evaluate the \emph{robustness} of the algorithm by analyzing its convergence properties. 
For synthetic data the ground truth shape parameters $\bar\beta$ are readily available, and the sphere-mesh model $\mathcal{M}(\theta,\beta)$ is animated in time with the $\bar\theta_n$ parameters of the complex motions in the \texttt{Handy/Teaser} dataset~\cite{tkach2016sphere}.
The model $\mathcal{M}(\theta,\beta)$ is rendered with the camera intrinsics producing a depth map $\mathcal{D}(\theta,\beta)$. We employ two metrics, one measuring average ground-truth residuals, the other measuring the average length of data-to-model ICP correspondences (i.e. part of the tracking energy)=:
% 
\begin{align}
E_{\beta} &= \tfrac{1}{|M|} \sum_{m \in M} \left| \beta_{[m]} - \bar\beta_{[m]} \right|
\\
E_\text{d2m}^n &= \tfrac{1}{|d_n|} \sum_{\mathbf{p}_j \in d_n} \left| \mathbf{p}_j - \Pi_{\mathcal{D}(\theta,\beta)}(\mathbf{p}_j) \right|_2
\label{eq:metrics}
\end{align}
% 
Where $\Pi$ is an operator computing the closest-point projection of point in the sensor's point cloud, onto the point-cloud associated with the synthetic depth-map $\mathcal{D}(\theta,\beta)$; see~\cite{tkach2016sphere}. For ground-truth comparisons, note how we selected only a subset \todo{$M=\{1,2,3,4\}$} of shape parameters; see \Figure{handmodel}. This is necessary as sphere-centres on the palm can move without affecting the tracking energy -- a null-space of our optimization.
% 
The tracking algorithm is initialized in the first frame by $\bar\theta_0$, while initializations for the user-personalized models are drawn from the uniform distribution {\small $\mathcal{U}(\bar\beta, \sigma)$}.
We limit ourselves to ten samples per value of $\sigma$, as each sample requires the re-execution of our algorithm on an entire sequence. Nonetheless, we empirically noticed that increasing the number of samples did not alter our observations.
% \AT{I still REALLY don't like this... It is not statistically significant. We'd run at least 50-100 samples...}
In \Figure{synthetic}, we visualize our two metrics as we vary the intensity of the perturbation. Our perturbation analysis is performed over \todo{2h30s} of synthetic footage, thus in our \VideoSynth{}, we only report a \emph{random selection} of tracked sequences.
% 

\subsection{Synthetic dataset: formulation analysis -- \Fig{interreal}}
\label{sec:analysis}
In \Figure{interreal}, we report an experiment analogous to that of \Figure{inter} but on a full 3D sequence. For example, consider the {\small \todo{$\beta_{[?]}$}} plot, where we report the runtime estimates for the length of the middle-finger's middle-phalanx (to avoid confusion with frame number $n$, the \todo{$[?]$} subscript is implied).
As expected, the phalanx length can be estimated only in those frames where the finger is bent, as the callout renderings show.
%
Notice how the intra-frame estimate {\small $\star\beta_n$} can oscillate away from the ground truth, and such incorrect estimates should be considered. To achieve this, our algorithm estimates the solution uncertainty {\small $\Sigma^*_n$}, and accordingly updates the online cumulative estimates {\small $\hat\beta_n$} and {\small $\hat\Sigma_n$}.

\providecommand{\off}{5}
\begin{figure}[t]
\centering
\begin{overpic} 
[width=\linewidth]
% [width=\linewidth,height=3in,grid,tics=10]
{\currfiledir/item.pdf}
\myfigurename{}
\put(97,1.5){\small $\varepsilon$}
% \put(96,45){\small $\varepsilon$}
\put(0,38){\scriptsize \vertical{$\% (max)$}}
\put(0,83){\scriptsize \vertical{$\% (avg)$}}
% ALGORITHMS LEGEND
\put(\off,43){\scriptsize \color[RGB]{77,77,77}      Scaled Template (anisotropic)}
\put(\off,40){\scriptsize \color[RGB]{197,151,53}    Batch-Offline -- \Sec{batch}}
\put(\off,37){\scriptsize \color[RGB]{160,215,190}   Batch-Online -- \Sec{batch}}
\put(\off,34){\scriptsize \color[RGB]{61,131,119}    Intra-frame -- \Eq{energies}}
\put(\off,31){\scriptsize \color[RGB]{217,144,143}   Split inter-frame (KF) -- \Tab{kf-like}}
\put(\off,28){\scriptsize \color[RGB]{178,68,117}    Joint inter-frame (EKF) -- \Tab{iekf-like}}
\put(\off,25){\scriptsize \color[RGB]{150,29,29}     \cite{taylor2016joint}}
\put(\off,22){\scriptsize \color[RGB]{30,150,30}     \cite{tompson2014real}}
\put(\off,19){\scriptsize \color[RGB]{150,149,30}    \cite{htrack}}
\put(\off,16){\scriptsize \color[RGB]{29,30,150}     \cite{sridhar2015fast}}
\put(\off,13){\scriptsize \color[RGB]{150,30,150}    \cite{oberweger2015hands}}
\put(\off,10){\scriptsize \color[RGB]{29,150,150}     \cite{tang2015opening}}
\put(\off,7){\scriptsize \color[RGB]{150,150,150}    \cite{tan2016fits}} 
\end{overpic}
% \vspace{-.3in}
\caption{
% 
Evaluation on the NYU dataset from \protect\cite{tompson2014real} reporting the percentage of frames with average (top) and max (bottom) ground-truth marker-error  less than $\varepsilon$.
% 
}
\label{fig:evalnyu}
\end{figure}

\subsection{Calibration dataset: \texttt{GuessWho?} -- \Fig{realtrack}}
\label{sec:evaldataset}
We stress-tested our system by qualitative evaluating our calibration technique with data acquired from \emph{twelve} different users performing in front of an Intel RealSense SR300 camera (time-of-flight depth sensor). 
Snapshot of the twelve calibration sequences are reported in \Figure{realtrack}, but the calibration procedure can be better appreciated by watching our \VideoQualitative{}.
While ground truth information is not available, these datasets can be evaluated through the use of empirical metrics; e.g. $E_\text{d2m}$ and $E_\text{m2d}$ from \cite{tkach2016sphere}, or the golden-energy from~\cite{taylor2016joint}.
Hence, to foster research in this domain, we release these calibration sequences to the community as the ``\texttt{GuessWho?}'' dataset -- future works will be able to generate identical synthetic clones of human hands (metrics~$\approx 0$), and thus enable biometric authentication applications.

\definecolor{truth}{RGB}{180,180,180}
\definecolor{intra}{RGB}{188,58,117}
\definecolor{inter}{RGB}{61,131,119}
\definecolor{varintra}{RGB}{227,176,171}
\definecolor{varinter}{RGB}{160,215,190}

% 
\begin{figure*}[t]
\centering
\begin{overpic} 
[width=\linewidth]
% [width=\linewidth,grid,tics=10]
{\currfiledir/item.pdf}
\myfigurename{}
\put(0.5,17){\scriptsize \vertical{$E_\beta$}}
\put(98,17){\scriptsize \vertical{$\Sigma^{-1}$}}
\put(6,-1.5){\small thumb middle length -- \todo{$\beta_{[?]}$}}
\put(31,-1.5){\small ring attachment y -- \todo{$\beta_{[?]}$}}
\put(54,-1.5){\small middle middle length -- \todo{$\beta_{[?]}$}}
\put(80,-1.5){\small middle attachment z -- \todo{$\beta_{[?]}$}}
\end{overpic}
\caption{
% 
We illustrate the formulation in \Section{technical} on the calibration of four different degrees-of-freedom on the \texttt{Handy/Teaser} dataset. 
(top)
% 
Frames are indexed by $n$ and we display: 
{\color{truth} ground-truth value $\bar{\beta}$},
{\color{intra} intra-frame estimate $\star{\beta}_n$} and 
{\color{inter} inter-frame estimate $\hat{\beta}_n$}; the scale for these quantities is relative to $\bar{\beta}$, and shown on left-hand side of each axis.
% 
In the same plot, we also visualize the \emph{inverse} of the diagonal entry of the covariance matrices (i.e. certainty) for {\color{varintra} intra-frame $\star\Sigma_n$} and {\color{varinter} inter-frame $\hat\Sigma_n$} estimates; the scale for these quantities is on the right-hand side of each plot.
% 
(bottom) We also display the pose corresponding to a selection of frames in the sequence~(small grey arrows); please note the correlation between pose and estimate uncertainty.
% 
\AT{might not be a bad idea to only have 3x figures here? Thoughts?}
% 
}
\label{fig:interreal}
\end{figure*}

\renewcommand{\off}{6}
\begin{figure}[b]
\centering
\begin{overpic} 
[width=\linewidth]
% [width=\linewidth,height=1.5in,grid,tics=10]
{\currfiledir/item.pdf}
\put(97,1.5){\small $\varepsilon$}
\put(0,38){\scriptsize \vertical{$\% (avg)$}}
\myfigurename{}
\put(\off,42){\scriptsize \color[RGB]{61,131,119}      Intra-frame -- \Eq{energies}}
\put(\off,39){\scriptsize \color[RGB]{219,158,148}     Split inter-frame (KF) -- \Tab{kf-like}}
\put(\off,36){\scriptsize \color[RGB]{182,78,124}      Joint inter-frame (EKF) -- \Tab{iekf-like}}
\put(\off,33){\scriptsize \color[RGB]{150,149,30}     \cite{htrack}}
\put(\off,30){\scriptsize \color[RGB]{150,29,29}      \cite{taylor2016joint}}
\put(\off,27){\scriptsize \color[RGB]{129,76,145}      \cite{tkach2016sphere}}
\put(\off,24){\scriptsize \color[RGB]{100,100,100}      \cite{sharp2015accurate}}
\end{overpic}
\caption{
% 
Evaluation on the \texttt{Handy/Teaser} dataset, reporting the percentage of frames with an average $\tilde{E}_\text{d2m}$ data-to-model energy below $\varepsilon$.
% 
}
\label{fig:evalhandy}
\end{figure}

\subsection{State-of-the-art on $\{$\texttt{NYU, Handy}$\}$ --
\Fig{evalnyu} and \Fig{evalhandy}}
\label{sec:evalstar}
% We quantitatively evaluate our algorithm in terms of tracking performance on .
To quantitatively evaluate real-time tracking algorithms, while many \emph{marker-based} datasets are available (e.g. \cite{qian2014realtime}, \cite{sharp2015accurate} and \cite{yuan2017bighand}), state-of-the-art \emph{generative} methods have focused on the \texttt{NYU} \cite{tompson2014real} and \texttt{Handy} \cite{tkach2016sphere} datasets. 
These datasets allow us to compare our method and its variants to a number of other algorithms including: the PSO tracker by~\cite{sharp2015accurate}, the calibration methods by~\cite{khamis2015learning} and \cite{tan2016fits}, the subdivision tracker of~\cite{taylor2016joint}, the cylindrical tracker in~\cite{htrack}, the sphere-mesh calibration/tracker by~\cite{tkach2016sphere}, as well as the Gaussian tracker in~\cite{sridhar2015fast}, and finally discriminative methods such as~\cite{tompson2014real}, \cite{tang2015opening} and \cite{oberweger2015hands}. 
% 
On the \texttt{NYU} dataset, our algorithm achieves state-of-the-art performance, and is the first to be able to calibrate the user-personalized tracking model in real-time. Similarly to \cite{taylor2016joint}, we evaluate the metrics on the first user ($n \in [0, 2440]$), as the second is too far from the camera.
%
